
% Default to the notebook output style

    


% Inherit from the specified cell style.




    
\documentclass[11pt]{article}

    
    
    \usepackage[T1]{fontenc}
    % Nicer default font (+ math font) than Computer Modern for most use cases
    \usepackage{mathpazo}

    % Basic figure setup, for now with no caption control since it's done
    % automatically by Pandoc (which extracts ![](path) syntax from Markdown).
    \usepackage{graphicx}
    % We will generate all images so they have a width \maxwidth. This means
    % that they will get their normal width if they fit onto the page, but
    % are scaled down if they would overflow the margins.
    \makeatletter
    \def\maxwidth{\ifdim\Gin@nat@width>\linewidth\linewidth
    \else\Gin@nat@width\fi}
    \makeatother
    \let\Oldincludegraphics\includegraphics
    % Set max figure width to be 80% of text width, for now hardcoded.
    \renewcommand{\includegraphics}[1]{\Oldincludegraphics[width=.8\maxwidth]{#1}}
    % Ensure that by default, figures have no caption (until we provide a
    % proper Figure object with a Caption API and a way to capture that
    % in the conversion process - todo).
    \usepackage{caption}
    \DeclareCaptionLabelFormat{nolabel}{}
    \captionsetup{labelformat=nolabel}

    \usepackage{adjustbox} % Used to constrain images to a maximum size 
    \usepackage{xcolor} % Allow colors to be defined
    \usepackage{enumerate} % Needed for markdown enumerations to work
    \usepackage{geometry} % Used to adjust the document margins
    \usepackage{amsmath} % Equations
    \usepackage{amssymb} % Equations
    \usepackage{textcomp} % defines textquotesingle
    % Hack from http://tex.stackexchange.com/a/47451/13684:
    \AtBeginDocument{%
        \def\PYZsq{\textquotesingle}% Upright quotes in Pygmentized code
    }
    \usepackage{upquote} % Upright quotes for verbatim code
    \usepackage{eurosym} % defines \euro
    \usepackage[mathletters]{ucs} % Extended unicode (utf-8) support
    \usepackage[utf8x]{inputenc} % Allow utf-8 characters in the tex document
    \usepackage{fancyvrb} % verbatim replacement that allows latex
    \usepackage{grffile} % extends the file name processing of package graphics 
                         % to support a larger range 
    % The hyperref package gives us a pdf with properly built
    % internal navigation ('pdf bookmarks' for the table of contents,
    % internal cross-reference links, web links for URLs, etc.)
    \usepackage{hyperref}
    \usepackage{longtable} % longtable support required by pandoc >1.10
    \usepackage{booktabs}  % table support for pandoc > 1.12.2
    \usepackage[inline]{enumitem} % IRkernel/repr support (it uses the enumerate* environment)
    \usepackage[normalem]{ulem} % ulem is needed to support strikethroughs (\sout)
                                % normalem makes italics be italics, not underlines
    

    
    
    % Colors for the hyperref package
    \definecolor{urlcolor}{rgb}{0,.145,.698}
    \definecolor{linkcolor}{rgb}{.71,0.21,0.01}
    \definecolor{citecolor}{rgb}{.12,.54,.11}

    % ANSI colors
    \definecolor{ansi-black}{HTML}{3E424D}
    \definecolor{ansi-black-intense}{HTML}{282C36}
    \definecolor{ansi-red}{HTML}{E75C58}
    \definecolor{ansi-red-intense}{HTML}{B22B31}
    \definecolor{ansi-green}{HTML}{00A250}
    \definecolor{ansi-green-intense}{HTML}{007427}
    \definecolor{ansi-yellow}{HTML}{DDB62B}
    \definecolor{ansi-yellow-intense}{HTML}{B27D12}
    \definecolor{ansi-blue}{HTML}{208FFB}
    \definecolor{ansi-blue-intense}{HTML}{0065CA}
    \definecolor{ansi-magenta}{HTML}{D160C4}
    \definecolor{ansi-magenta-intense}{HTML}{A03196}
    \definecolor{ansi-cyan}{HTML}{60C6C8}
    \definecolor{ansi-cyan-intense}{HTML}{258F8F}
    \definecolor{ansi-white}{HTML}{C5C1B4}
    \definecolor{ansi-white-intense}{HTML}{A1A6B2}

    % commands and environments needed by pandoc snippets
    % extracted from the output of `pandoc -s`
    \providecommand{\tightlist}{%
      \setlength{\itemsep}{0pt}\setlength{\parskip}{0pt}}
    \DefineVerbatimEnvironment{Highlighting}{Verbatim}{commandchars=\\\{\}}
    % Add ',fontsize=\small' for more characters per line
    \newenvironment{Shaded}{}{}
    \newcommand{\KeywordTok}[1]{\textcolor[rgb]{0.00,0.44,0.13}{\textbf{{#1}}}}
    \newcommand{\DataTypeTok}[1]{\textcolor[rgb]{0.56,0.13,0.00}{{#1}}}
    \newcommand{\DecValTok}[1]{\textcolor[rgb]{0.25,0.63,0.44}{{#1}}}
    \newcommand{\BaseNTok}[1]{\textcolor[rgb]{0.25,0.63,0.44}{{#1}}}
    \newcommand{\FloatTok}[1]{\textcolor[rgb]{0.25,0.63,0.44}{{#1}}}
    \newcommand{\CharTok}[1]{\textcolor[rgb]{0.25,0.44,0.63}{{#1}}}
    \newcommand{\StringTok}[1]{\textcolor[rgb]{0.25,0.44,0.63}{{#1}}}
    \newcommand{\CommentTok}[1]{\textcolor[rgb]{0.38,0.63,0.69}{\textit{{#1}}}}
    \newcommand{\OtherTok}[1]{\textcolor[rgb]{0.00,0.44,0.13}{{#1}}}
    \newcommand{\AlertTok}[1]{\textcolor[rgb]{1.00,0.00,0.00}{\textbf{{#1}}}}
    \newcommand{\FunctionTok}[1]{\textcolor[rgb]{0.02,0.16,0.49}{{#1}}}
    \newcommand{\RegionMarkerTok}[1]{{#1}}
    \newcommand{\ErrorTok}[1]{\textcolor[rgb]{1.00,0.00,0.00}{\textbf{{#1}}}}
    \newcommand{\NormalTok}[1]{{#1}}
    
    % Additional commands for more recent versions of Pandoc
    \newcommand{\ConstantTok}[1]{\textcolor[rgb]{0.53,0.00,0.00}{{#1}}}
    \newcommand{\SpecialCharTok}[1]{\textcolor[rgb]{0.25,0.44,0.63}{{#1}}}
    \newcommand{\VerbatimStringTok}[1]{\textcolor[rgb]{0.25,0.44,0.63}{{#1}}}
    \newcommand{\SpecialStringTok}[1]{\textcolor[rgb]{0.73,0.40,0.53}{{#1}}}
    \newcommand{\ImportTok}[1]{{#1}}
    \newcommand{\DocumentationTok}[1]{\textcolor[rgb]{0.73,0.13,0.13}{\textit{{#1}}}}
    \newcommand{\AnnotationTok}[1]{\textcolor[rgb]{0.38,0.63,0.69}{\textbf{\textit{{#1}}}}}
    \newcommand{\CommentVarTok}[1]{\textcolor[rgb]{0.38,0.63,0.69}{\textbf{\textit{{#1}}}}}
    \newcommand{\VariableTok}[1]{\textcolor[rgb]{0.10,0.09,0.49}{{#1}}}
    \newcommand{\ControlFlowTok}[1]{\textcolor[rgb]{0.00,0.44,0.13}{\textbf{{#1}}}}
    \newcommand{\OperatorTok}[1]{\textcolor[rgb]{0.40,0.40,0.40}{{#1}}}
    \newcommand{\BuiltInTok}[1]{{#1}}
    \newcommand{\ExtensionTok}[1]{{#1}}
    \newcommand{\PreprocessorTok}[1]{\textcolor[rgb]{0.74,0.48,0.00}{{#1}}}
    \newcommand{\AttributeTok}[1]{\textcolor[rgb]{0.49,0.56,0.16}{{#1}}}
    \newcommand{\InformationTok}[1]{\textcolor[rgb]{0.38,0.63,0.69}{\textbf{\textit{{#1}}}}}
    \newcommand{\WarningTok}[1]{\textcolor[rgb]{0.38,0.63,0.69}{\textbf{\textit{{#1}}}}}
    
    
    % Define a nice break command that doesn't care if a line doesn't already
    % exist.
    \def\br{\hspace*{\fill} \\* }
    % Math Jax compatability definitions
    \def\gt{>}
    \def\lt{<}
    % Document parameters
    \title{WashburnPaulHW1}
    
    
    

    % Pygments definitions
    
\makeatletter
\def\PY@reset{\let\PY@it=\relax \let\PY@bf=\relax%
    \let\PY@ul=\relax \let\PY@tc=\relax%
    \let\PY@bc=\relax \let\PY@ff=\relax}
\def\PY@tok#1{\csname PY@tok@#1\endcsname}
\def\PY@toks#1+{\ifx\relax#1\empty\else%
    \PY@tok{#1}\expandafter\PY@toks\fi}
\def\PY@do#1{\PY@bc{\PY@tc{\PY@ul{%
    \PY@it{\PY@bf{\PY@ff{#1}}}}}}}
\def\PY#1#2{\PY@reset\PY@toks#1+\relax+\PY@do{#2}}

\expandafter\def\csname PY@tok@kd\endcsname{\let\PY@bf=\textbf\def\PY@tc##1{\textcolor[rgb]{0.00,0.50,0.00}{##1}}}
\expandafter\def\csname PY@tok@sc\endcsname{\def\PY@tc##1{\textcolor[rgb]{0.73,0.13,0.13}{##1}}}
\expandafter\def\csname PY@tok@sr\endcsname{\def\PY@tc##1{\textcolor[rgb]{0.73,0.40,0.53}{##1}}}
\expandafter\def\csname PY@tok@il\endcsname{\def\PY@tc##1{\textcolor[rgb]{0.40,0.40,0.40}{##1}}}
\expandafter\def\csname PY@tok@ow\endcsname{\let\PY@bf=\textbf\def\PY@tc##1{\textcolor[rgb]{0.67,0.13,1.00}{##1}}}
\expandafter\def\csname PY@tok@fm\endcsname{\def\PY@tc##1{\textcolor[rgb]{0.00,0.00,1.00}{##1}}}
\expandafter\def\csname PY@tok@err\endcsname{\def\PY@bc##1{\setlength{\fboxsep}{0pt}\fcolorbox[rgb]{1.00,0.00,0.00}{1,1,1}{\strut ##1}}}
\expandafter\def\csname PY@tok@s\endcsname{\def\PY@tc##1{\textcolor[rgb]{0.73,0.13,0.13}{##1}}}
\expandafter\def\csname PY@tok@mf\endcsname{\def\PY@tc##1{\textcolor[rgb]{0.40,0.40,0.40}{##1}}}
\expandafter\def\csname PY@tok@mi\endcsname{\def\PY@tc##1{\textcolor[rgb]{0.40,0.40,0.40}{##1}}}
\expandafter\def\csname PY@tok@no\endcsname{\def\PY@tc##1{\textcolor[rgb]{0.53,0.00,0.00}{##1}}}
\expandafter\def\csname PY@tok@mh\endcsname{\def\PY@tc##1{\textcolor[rgb]{0.40,0.40,0.40}{##1}}}
\expandafter\def\csname PY@tok@nv\endcsname{\def\PY@tc##1{\textcolor[rgb]{0.10,0.09,0.49}{##1}}}
\expandafter\def\csname PY@tok@ne\endcsname{\let\PY@bf=\textbf\def\PY@tc##1{\textcolor[rgb]{0.82,0.25,0.23}{##1}}}
\expandafter\def\csname PY@tok@cs\endcsname{\let\PY@it=\textit\def\PY@tc##1{\textcolor[rgb]{0.25,0.50,0.50}{##1}}}
\expandafter\def\csname PY@tok@nl\endcsname{\def\PY@tc##1{\textcolor[rgb]{0.63,0.63,0.00}{##1}}}
\expandafter\def\csname PY@tok@o\endcsname{\def\PY@tc##1{\textcolor[rgb]{0.40,0.40,0.40}{##1}}}
\expandafter\def\csname PY@tok@sh\endcsname{\def\PY@tc##1{\textcolor[rgb]{0.73,0.13,0.13}{##1}}}
\expandafter\def\csname PY@tok@kr\endcsname{\let\PY@bf=\textbf\def\PY@tc##1{\textcolor[rgb]{0.00,0.50,0.00}{##1}}}
\expandafter\def\csname PY@tok@sb\endcsname{\def\PY@tc##1{\textcolor[rgb]{0.73,0.13,0.13}{##1}}}
\expandafter\def\csname PY@tok@vm\endcsname{\def\PY@tc##1{\textcolor[rgb]{0.10,0.09,0.49}{##1}}}
\expandafter\def\csname PY@tok@gd\endcsname{\def\PY@tc##1{\textcolor[rgb]{0.63,0.00,0.00}{##1}}}
\expandafter\def\csname PY@tok@cpf\endcsname{\let\PY@it=\textit\def\PY@tc##1{\textcolor[rgb]{0.25,0.50,0.50}{##1}}}
\expandafter\def\csname PY@tok@cm\endcsname{\let\PY@it=\textit\def\PY@tc##1{\textcolor[rgb]{0.25,0.50,0.50}{##1}}}
\expandafter\def\csname PY@tok@dl\endcsname{\def\PY@tc##1{\textcolor[rgb]{0.73,0.13,0.13}{##1}}}
\expandafter\def\csname PY@tok@gr\endcsname{\def\PY@tc##1{\textcolor[rgb]{1.00,0.00,0.00}{##1}}}
\expandafter\def\csname PY@tok@m\endcsname{\def\PY@tc##1{\textcolor[rgb]{0.40,0.40,0.40}{##1}}}
\expandafter\def\csname PY@tok@nf\endcsname{\def\PY@tc##1{\textcolor[rgb]{0.00,0.00,1.00}{##1}}}
\expandafter\def\csname PY@tok@nd\endcsname{\def\PY@tc##1{\textcolor[rgb]{0.67,0.13,1.00}{##1}}}
\expandafter\def\csname PY@tok@ge\endcsname{\let\PY@it=\textit}
\expandafter\def\csname PY@tok@k\endcsname{\let\PY@bf=\textbf\def\PY@tc##1{\textcolor[rgb]{0.00,0.50,0.00}{##1}}}
\expandafter\def\csname PY@tok@vc\endcsname{\def\PY@tc##1{\textcolor[rgb]{0.10,0.09,0.49}{##1}}}
\expandafter\def\csname PY@tok@si\endcsname{\let\PY@bf=\textbf\def\PY@tc##1{\textcolor[rgb]{0.73,0.40,0.53}{##1}}}
\expandafter\def\csname PY@tok@gi\endcsname{\def\PY@tc##1{\textcolor[rgb]{0.00,0.63,0.00}{##1}}}
\expandafter\def\csname PY@tok@gs\endcsname{\let\PY@bf=\textbf}
\expandafter\def\csname PY@tok@gp\endcsname{\let\PY@bf=\textbf\def\PY@tc##1{\textcolor[rgb]{0.00,0.00,0.50}{##1}}}
\expandafter\def\csname PY@tok@vg\endcsname{\def\PY@tc##1{\textcolor[rgb]{0.10,0.09,0.49}{##1}}}
\expandafter\def\csname PY@tok@kc\endcsname{\let\PY@bf=\textbf\def\PY@tc##1{\textcolor[rgb]{0.00,0.50,0.00}{##1}}}
\expandafter\def\csname PY@tok@nc\endcsname{\let\PY@bf=\textbf\def\PY@tc##1{\textcolor[rgb]{0.00,0.00,1.00}{##1}}}
\expandafter\def\csname PY@tok@kt\endcsname{\def\PY@tc##1{\textcolor[rgb]{0.69,0.00,0.25}{##1}}}
\expandafter\def\csname PY@tok@gt\endcsname{\def\PY@tc##1{\textcolor[rgb]{0.00,0.27,0.87}{##1}}}
\expandafter\def\csname PY@tok@c\endcsname{\let\PY@it=\textit\def\PY@tc##1{\textcolor[rgb]{0.25,0.50,0.50}{##1}}}
\expandafter\def\csname PY@tok@ss\endcsname{\def\PY@tc##1{\textcolor[rgb]{0.10,0.09,0.49}{##1}}}
\expandafter\def\csname PY@tok@mb\endcsname{\def\PY@tc##1{\textcolor[rgb]{0.40,0.40,0.40}{##1}}}
\expandafter\def\csname PY@tok@go\endcsname{\def\PY@tc##1{\textcolor[rgb]{0.53,0.53,0.53}{##1}}}
\expandafter\def\csname PY@tok@na\endcsname{\def\PY@tc##1{\textcolor[rgb]{0.49,0.56,0.16}{##1}}}
\expandafter\def\csname PY@tok@sa\endcsname{\def\PY@tc##1{\textcolor[rgb]{0.73,0.13,0.13}{##1}}}
\expandafter\def\csname PY@tok@bp\endcsname{\def\PY@tc##1{\textcolor[rgb]{0.00,0.50,0.00}{##1}}}
\expandafter\def\csname PY@tok@gh\endcsname{\let\PY@bf=\textbf\def\PY@tc##1{\textcolor[rgb]{0.00,0.00,0.50}{##1}}}
\expandafter\def\csname PY@tok@kn\endcsname{\let\PY@bf=\textbf\def\PY@tc##1{\textcolor[rgb]{0.00,0.50,0.00}{##1}}}
\expandafter\def\csname PY@tok@mo\endcsname{\def\PY@tc##1{\textcolor[rgb]{0.40,0.40,0.40}{##1}}}
\expandafter\def\csname PY@tok@w\endcsname{\def\PY@tc##1{\textcolor[rgb]{0.73,0.73,0.73}{##1}}}
\expandafter\def\csname PY@tok@cp\endcsname{\def\PY@tc##1{\textcolor[rgb]{0.74,0.48,0.00}{##1}}}
\expandafter\def\csname PY@tok@nt\endcsname{\let\PY@bf=\textbf\def\PY@tc##1{\textcolor[rgb]{0.00,0.50,0.00}{##1}}}
\expandafter\def\csname PY@tok@nn\endcsname{\let\PY@bf=\textbf\def\PY@tc##1{\textcolor[rgb]{0.00,0.00,1.00}{##1}}}
\expandafter\def\csname PY@tok@ni\endcsname{\let\PY@bf=\textbf\def\PY@tc##1{\textcolor[rgb]{0.60,0.60,0.60}{##1}}}
\expandafter\def\csname PY@tok@nb\endcsname{\def\PY@tc##1{\textcolor[rgb]{0.00,0.50,0.00}{##1}}}
\expandafter\def\csname PY@tok@ch\endcsname{\let\PY@it=\textit\def\PY@tc##1{\textcolor[rgb]{0.25,0.50,0.50}{##1}}}
\expandafter\def\csname PY@tok@kp\endcsname{\def\PY@tc##1{\textcolor[rgb]{0.00,0.50,0.00}{##1}}}
\expandafter\def\csname PY@tok@c1\endcsname{\let\PY@it=\textit\def\PY@tc##1{\textcolor[rgb]{0.25,0.50,0.50}{##1}}}
\expandafter\def\csname PY@tok@sx\endcsname{\def\PY@tc##1{\textcolor[rgb]{0.00,0.50,0.00}{##1}}}
\expandafter\def\csname PY@tok@s2\endcsname{\def\PY@tc##1{\textcolor[rgb]{0.73,0.13,0.13}{##1}}}
\expandafter\def\csname PY@tok@vi\endcsname{\def\PY@tc##1{\textcolor[rgb]{0.10,0.09,0.49}{##1}}}
\expandafter\def\csname PY@tok@gu\endcsname{\let\PY@bf=\textbf\def\PY@tc##1{\textcolor[rgb]{0.50,0.00,0.50}{##1}}}
\expandafter\def\csname PY@tok@s1\endcsname{\def\PY@tc##1{\textcolor[rgb]{0.73,0.13,0.13}{##1}}}
\expandafter\def\csname PY@tok@sd\endcsname{\let\PY@it=\textit\def\PY@tc##1{\textcolor[rgb]{0.73,0.13,0.13}{##1}}}
\expandafter\def\csname PY@tok@se\endcsname{\let\PY@bf=\textbf\def\PY@tc##1{\textcolor[rgb]{0.73,0.40,0.13}{##1}}}

\def\PYZbs{\char`\\}
\def\PYZus{\char`\_}
\def\PYZob{\char`\{}
\def\PYZcb{\char`\}}
\def\PYZca{\char`\^}
\def\PYZam{\char`\&}
\def\PYZlt{\char`\<}
\def\PYZgt{\char`\>}
\def\PYZsh{\char`\#}
\def\PYZpc{\char`\%}
\def\PYZdl{\char`\$}
\def\PYZhy{\char`\-}
\def\PYZsq{\char`\'}
\def\PYZdq{\char`\"}
\def\PYZti{\char`\~}
% for compatibility with earlier versions
\def\PYZat{@}
\def\PYZlb{[}
\def\PYZrb{]}
\makeatother


    % Exact colors from NB
    \definecolor{incolor}{rgb}{0.0, 0.0, 0.5}
    \definecolor{outcolor}{rgb}{0.545, 0.0, 0.0}



    
    % Prevent overflowing lines due to hard-to-break entities
    \sloppy 
    % Setup hyperref package
    \hypersetup{
      breaklinks=true,  % so long urls are correctly broken across lines
      colorlinks=true,
      urlcolor=urlcolor,
      linkcolor=linkcolor,
      citecolor=citecolor,
      }
    % Slightly bigger margins than the latex defaults
    
    \geometry{verbose,tmargin=1in,bmargin=1in,lmargin=1in,rmargin=1in}
    
    

    \begin{document}
    
    
    \maketitle
    
    

    
    \#\#

CSCI E-82

\#\#

HW 1 Dimensionality Reduction

\#\#\#

Due: Sept 17, 2018 11:59pm EST

\paragraph{Note that this is an individual homework to be completed
without collaborations except through
Piazza.}\label{note-that-this-is-an-individual-homework-to-be-completed-without-collaborations-except-through-piazza.}

\paragraph{We encourage you to make progress this weekend since the
second homework will likely come out in a week before this one is
due.}\label{we-encourage-you-to-make-progress-this-weekend-since-the-second-homework-will-likely-come-out-in-a-week-before-this-one-is-due.}

    \subsubsection{Your name:}\label{your-name}

    PAUL M. WASHBURN

    \begin{Verbatim}[commandchars=\\\{\}]
{\color{incolor}In [{\color{incolor}1}]:} \PY{k+kn}{import} \PY{n+nn}{numpy} \PY{k}{as} \PY{n+nn}{np}
        \PY{k+kn}{import} \PY{n+nn}{pandas} \PY{k}{as} \PY{n+nn}{pd}
        \PY{k+kn}{import} \PY{n+nn}{matplotlib}\PY{n+nn}{.}\PY{n+nn}{pyplot} \PY{k}{as} \PY{n+nn}{plt}
        \PY{k+kn}{import} \PY{n+nn}{seaborn} \PY{k}{as} \PY{n+nn}{sns}
        \PY{k+kn}{import} \PY{n+nn}{random}
        \PY{k+kn}{from} \PY{n+nn}{sklearn}\PY{n+nn}{.}\PY{n+nn}{decomposition} \PY{k}{import} \PY{n}{PCA}
        \PY{o}{\PYZpc{}}\PY{k}{matplotlib} inline
        \PY{k+kn}{from} \PY{n+nn}{functools} \PY{k}{import} \PY{n}{wraps}
        \PY{k+kn}{import} \PY{n+nn}{time}
        \PY{k+kn}{from} \PY{n+nn}{sklearn}\PY{n+nn}{.}\PY{n+nn}{manifold} \PY{k}{import} \PY{n}{MDS}
        \PY{k+kn}{from} \PY{n+nn}{sklearn}\PY{n+nn}{.}\PY{n+nn}{manifold} \PY{k}{import} \PY{n}{TSNE}
        
        \PY{k}{def} \PY{n+nf}{timing\PYZus{}function}\PY{p}{(}\PY{n}{some\PYZus{}function}\PY{p}{)}\PY{p}{:}
            \PY{l+s+sd}{\PYZsq{}\PYZsq{}\PYZsq{}}
        \PY{l+s+sd}{    Decorator function.  Outputs the time a function takes to execute.}
        \PY{l+s+sd}{    \PYZsq{}\PYZsq{}\PYZsq{}}
            \PY{n+nd}{@wraps}\PY{p}{(}\PY{n}{some\PYZus{}function}\PY{p}{)}
            \PY{k}{def} \PY{n+nf}{wrapper}\PY{p}{(}\PY{o}{*}\PY{n}{args}\PY{p}{,} \PY{o}{*}\PY{o}{*}\PY{n}{kwargs}\PY{p}{)}\PY{p}{:}
                \PY{n}{t1} \PY{o}{=} \PY{n}{time}\PY{o}{.}\PY{n}{time}\PY{p}{(}\PY{p}{)}
                \PY{n}{result} \PY{o}{=} \PY{n}{some\PYZus{}function}\PY{p}{(}\PY{o}{*}\PY{n}{args}\PY{p}{,} \PY{o}{*}\PY{o}{*}\PY{n}{kwargs}\PY{p}{)}
                \PY{n}{t2} \PY{o}{=} \PY{n}{time}\PY{o}{.}\PY{n}{time}\PY{p}{(}\PY{p}{)}
                \PY{n}{time\PYZus{}elapsed} \PY{o}{=} \PY{n+nb}{round}\PY{p}{(}\PY{p}{(}\PY{n}{t2} \PY{o}{\PYZhy{}} \PY{n}{t1}\PY{p}{)}\PY{p}{,} \PY{l+m+mi}{2}\PY{p}{)}
                \PY{n+nb}{print}\PY{p}{(}\PY{l+s+s1}{\PYZsq{}}\PY{l+s+s1}{Runtime: }\PY{l+s+s1}{\PYZsq{}} \PY{o}{+} \PY{n+nb}{str}\PY{p}{(}\PY{n}{time\PYZus{}elapsed}\PY{p}{)} \PY{o}{+} \PY{l+s+s1}{\PYZsq{}}\PY{l+s+s1}{ seconds}\PY{l+s+s1}{\PYZsq{}}\PY{p}{)}
                \PY{k}{return} \PY{n}{result}
            
            \PY{k}{return} \PY{n}{wrapper}
\end{Verbatim}


    \subsubsection{Problem 1 (5 points)}\label{problem-1-5-points}

\[\mathbf{X} = \left[\begin{array}
{rrr}
1 & 2 & 3 \\
4 & 5 & 6 \\
7 & 8 & 9
\end{array}\right]
\]

\[\mathbf{Y} = \left[\begin{array}
{rrr}
1 & 2 & 1 \\
2 & 1 & 2  
\end{array}\right]
\]

Compute XYT. The answer can be computed by hand and written in Markdown
like the above matrices, or computed in python. Either way is
acceptable.

    \begin{Verbatim}[commandchars=\\\{\}]
{\color{incolor}In [{\color{incolor}2}]:} \PY{n}{X} \PY{o}{=} \PY{n}{np}\PY{o}{.}\PY{n}{arange}\PY{p}{(}\PY{l+m+mi}{1}\PY{p}{,} \PY{l+m+mi}{10}\PY{p}{)}\PY{o}{.}\PY{n}{reshape}\PY{p}{(}\PY{l+m+mi}{3}\PY{p}{,} \PY{l+m+mi}{3}\PY{p}{)}
        \PY{n}{Y} \PY{o}{=} \PY{n}{np}\PY{o}{.}\PY{n}{array}\PY{p}{(}\PY{p}{[}\PY{p}{[}\PY{l+m+mi}{1}\PY{p}{,} \PY{l+m+mi}{2}\PY{p}{,} \PY{l+m+mi}{1}\PY{p}{]}\PY{p}{,} \PY{p}{[}\PY{l+m+mi}{2}\PY{p}{,} \PY{l+m+mi}{1}\PY{p}{,} \PY{l+m+mi}{2}\PY{p}{]}\PY{p}{]}\PY{p}{)}
        \PY{n}{np}\PY{o}{.}\PY{n}{matmul}\PY{p}{(}\PY{n}{X}\PY{p}{,} \PY{n}{Y}\PY{o}{.}\PY{n}{T}\PY{p}{)}
\end{Verbatim}


\begin{Verbatim}[commandchars=\\\{\}]
{\color{outcolor}Out[{\color{outcolor}2}]:} array([[ 8, 10],
               [20, 25],
               [32, 40]])
\end{Verbatim}
            
    \begin{Verbatim}[commandchars=\\\{\}]
{\color{incolor}In [{\color{incolor}3}]:} \PY{n}{np}\PY{o}{.}\PY{n}{dot}\PY{p}{(}\PY{n}{X}\PY{p}{,} \PY{n}{Y}\PY{o}{.}\PY{n}{T}\PY{p}{)}
\end{Verbatim}


\begin{Verbatim}[commandchars=\\\{\}]
{\color{outcolor}Out[{\color{outcolor}3}]:} array([[ 8, 10],
               [20, 25],
               [32, 40]])
\end{Verbatim}
            
    \subsubsection{Problem 2}\label{problem-2}

This problem goes through a combination of python data manipulations as
well as the full math projection using PCA. We have divided the problem
into multiple parts.

    \subsubsection{Problem 2a (5 points)}\label{problem-2a-5-points}

Download and load in the data set from the UCI archive
https://archive.ics.uci.edu/ml/machine-learning-databases/ecoli/. Print
the dimensions and the first few rows to demonstrate a successful load.

    \begin{Verbatim}[commandchars=\\\{\}]
{\color{incolor}In [{\color{incolor}4}]:} \PY{c+c1}{\PYZsh{} read data from source}
        \PY{n}{url} \PY{o}{=} \PY{l+s+s1}{\PYZsq{}}\PY{l+s+s1}{https://archive.ics.uci.edu/ml/machine\PYZhy{}learning\PYZhy{}databases/ecoli/ecoli.data}\PY{l+s+s1}{\PYZsq{}}
        \PY{n}{colnames} \PY{o}{=} \PY{p}{[}\PY{l+s+s1}{\PYZsq{}}\PY{l+s+s1}{sequence\PYZus{}name}\PY{l+s+s1}{\PYZsq{}}\PY{p}{,} \PY{l+s+s1}{\PYZsq{}}\PY{l+s+s1}{mcg}\PY{l+s+s1}{\PYZsq{}}\PY{p}{,} \PY{l+s+s1}{\PYZsq{}}\PY{l+s+s1}{gvh}\PY{l+s+s1}{\PYZsq{}}\PY{p}{,} \PY{l+s+s1}{\PYZsq{}}\PY{l+s+s1}{lip}\PY{l+s+s1}{\PYZsq{}}\PY{p}{,} \PY{l+s+s1}{\PYZsq{}}\PY{l+s+s1}{chg}\PY{l+s+s1}{\PYZsq{}}\PY{p}{,} \PY{l+s+s1}{\PYZsq{}}\PY{l+s+s1}{aac}\PY{l+s+s1}{\PYZsq{}}\PY{p}{,} \PY{l+s+s1}{\PYZsq{}}\PY{l+s+s1}{alm1}\PY{l+s+s1}{\PYZsq{}}\PY{p}{,} \PY{l+s+s1}{\PYZsq{}}\PY{l+s+s1}{alm2}\PY{l+s+s1}{\PYZsq{}}\PY{p}{,} \PY{l+s+s1}{\PYZsq{}}\PY{l+s+s1}{target}\PY{l+s+s1}{\PYZsq{}}\PY{p}{]}
        \PY{n}{df} \PY{o}{=} \PY{n}{pd}\PY{o}{.}\PY{n}{read\PYZus{}fwf}\PY{p}{(}\PY{n}{url}\PY{p}{,} \PY{n}{header}\PY{o}{=}\PY{k+kc}{None}\PY{p}{)}
        \PY{n}{df}\PY{o}{.}\PY{n}{columns} \PY{o}{=} \PY{n}{colnames}
        \PY{n+nb}{print}\PY{p}{(}\PY{n}{df}\PY{o}{.}\PY{n}{head}\PY{p}{(}\PY{p}{)}\PY{p}{)}
        
        \PY{c+c1}{\PYZsh{} download}
        \PY{n}{df}\PY{o}{.}\PY{n}{to\PYZus{}csv}\PY{p}{(}\PY{l+s+s1}{\PYZsq{}}\PY{l+s+s1}{data/ecoli\PYZus{}data.csv}\PY{l+s+s1}{\PYZsq{}}\PY{p}{,} \PY{n}{index}\PY{o}{=}\PY{k+kc}{False}\PY{p}{)}
\end{Verbatim}


    \begin{Verbatim}[commandchars=\\\{\}]
  sequence\_name   mcg   gvh   lip  chg   aac  alm1  alm2 target
0     AAT\_ECOLI  0.49  0.29  0.48  0.5  0.56  0.24  0.35     cp
1    ACEA\_ECOLI  0.07  0.40  0.48  0.5  0.54  0.35  0.44     cp
2    ACEK\_ECOLI  0.56  0.40  0.48  0.5  0.49  0.37  0.46     cp
3    ACKA\_ECOLI  0.59  0.49  0.48  0.5  0.52  0.45  0.36     cp
4     ADI\_ECOLI  0.23  0.32  0.48  0.5  0.55  0.25  0.35     cp

    \end{Verbatim}

    \begin{Verbatim}[commandchars=\\\{\}]
{\color{incolor}In [{\color{incolor}5}]:} \PY{c+c1}{\PYZsh{} load}
        \PY{n}{df} \PY{o}{=} \PY{n}{pd}\PY{o}{.}\PY{n}{read\PYZus{}csv}\PY{p}{(}\PY{l+s+s1}{\PYZsq{}}\PY{l+s+s1}{data/ecoli\PYZus{}data.csv}\PY{l+s+s1}{\PYZsq{}}\PY{p}{)}
        
        \PY{c+c1}{\PYZsh{} describe dimensions}
        \PY{n+nb}{print}\PY{p}{(}\PY{l+s+s1}{\PYZsq{}}\PY{l+s+s1}{rows = }\PY{l+s+si}{\PYZpc{}i}\PY{l+s+s1}{, columns = }\PY{l+s+si}{\PYZpc{}i}\PY{l+s+s1}{\PYZsq{}} \PY{o}{\PYZpc{}}\PY{p}{(}\PY{n}{df}\PY{o}{.}\PY{n}{shape}\PY{p}{[}\PY{l+m+mi}{0}\PY{p}{]}\PY{p}{,} \PY{n}{df}\PY{o}{.}\PY{n}{shape}\PY{p}{[}\PY{l+m+mi}{1}\PY{p}{]}\PY{p}{)}\PY{p}{)}
        
        \PY{c+c1}{\PYZsh{} first five }
        \PY{n}{df}\PY{o}{.}\PY{n}{head}\PY{p}{(}\PY{p}{)}
\end{Verbatim}


    \begin{Verbatim}[commandchars=\\\{\}]
rows = 336, columns = 9

    \end{Verbatim}

\begin{Verbatim}[commandchars=\\\{\}]
{\color{outcolor}Out[{\color{outcolor}5}]:}   sequence\_name   mcg   gvh   lip  chg   aac  alm1  alm2 target
        0     AAT\_ECOLI  0.49  0.29  0.48  0.5  0.56  0.24  0.35     cp
        1    ACEA\_ECOLI  0.07  0.40  0.48  0.5  0.54  0.35  0.44     cp
        2    ACEK\_ECOLI  0.56  0.40  0.48  0.5  0.49  0.37  0.46     cp
        3    ACKA\_ECOLI  0.59  0.49  0.48  0.5  0.52  0.45  0.36     cp
        4     ADI\_ECOLI  0.23  0.32  0.48  0.5  0.55  0.25  0.35     cp
\end{Verbatim}
            
    \subsubsection{Problem 2b (10 points)}\label{problem-2b-10-points}

Compute and print the covariance matrix for all columns excluding the
first and last. Rather than use the built-in function, compute this
using python code for practice. The following equation will suffice for
this.

Cov(X, Y) = Σ ( Xi - X ) ( Yi - Y ) / N

    \begin{Verbatim}[commandchars=\\\{\}]
{\color{incolor}In [{\color{incolor}6}]:} \PY{k}{def} \PY{n+nf}{cov\PYZus{}xy}\PY{p}{(}\PY{n}{m}\PY{p}{,} \PY{n}{y}\PY{p}{)}\PY{p}{:}
            \PY{l+s+sd}{\PYZsq{}\PYZsq{}\PYZsq{}}
        \PY{l+s+sd}{    Input vectors as 1D arrays}
        \PY{l+s+sd}{    \PYZsq{}\PYZsq{}\PYZsq{}}
            \PY{n}{X} \PY{o}{=} \PY{n}{np}\PY{o}{.}\PY{n}{concatenate}\PY{p}{(}\PY{p}{(}\PY{n}{m}\PY{p}{[}\PY{p}{:}\PY{p}{,}\PY{n}{np}\PY{o}{.}\PY{n}{newaxis}\PY{p}{]}\PY{p}{,} \PY{n}{y}\PY{p}{[}\PY{p}{:}\PY{p}{,}\PY{n}{np}\PY{o}{.}\PY{n}{newaxis}\PY{p}{]}\PY{p}{)}\PY{p}{,} \PY{n}{axis}\PY{o}{=}\PY{l+m+mi}{1}\PY{p}{)}
            \PY{n}{N} \PY{o}{=} \PY{n}{X}\PY{o}{.}\PY{n}{shape}\PY{p}{[}\PY{l+m+mi}{0}\PY{p}{]} \PY{o}{\PYZhy{}} \PY{l+m+mi}{1}
            \PY{n}{X} \PY{o}{\PYZhy{}}\PY{o}{=} \PY{n}{X}\PY{o}{.}\PY{n}{mean}\PY{p}{(}\PY{n}{axis}\PY{o}{=}\PY{l+m+mi}{0}\PY{p}{)}
            \PY{n}{C} \PY{o}{=} \PY{p}{(}\PY{n}{np}\PY{o}{.}\PY{n}{dot}\PY{p}{(}\PY{n}{X}\PY{o}{.}\PY{n}{T}\PY{p}{,} \PY{n}{X}\PY{o}{.}\PY{n}{conj}\PY{p}{(}\PY{p}{)}\PY{p}{)} \PY{o}{/} \PY{n}{N}\PY{p}{)}\PY{o}{.}\PY{n}{squeeze}\PY{p}{(}\PY{p}{)}
            \PY{k}{return} \PY{n}{C}
        
        \PY{k}{def} \PY{n+nf}{cov\PYZus{}matrix}\PY{p}{(}\PY{n}{df}\PY{p}{,} \PY{n}{cols}\PY{p}{)}\PY{p}{:}
            \PY{l+s+sd}{\PYZsq{}\PYZsq{}\PYZsq{}}
        \PY{l+s+sd}{    Returns covariance matrix with column names}
        \PY{l+s+sd}{    \PYZsq{}\PYZsq{}\PYZsq{}}
            \PY{c+c1}{\PYZsh{} create empty dataframe with variance on the diagonals}
            \PY{c+c1}{\PYZsh{} while preserving names \PYZhy{}\PYZhy{} leverages pd.DataFrame object}
            \PY{c+c1}{\PYZsh{} df.var() method}
            \PY{n}{cov\PYZus{}df} \PY{o}{=} \PY{n}{pd}\PY{o}{.}\PY{n}{DataFrame}\PY{p}{(}\PY{n}{np}\PY{o}{.}\PY{n}{diag}\PY{p}{(}\PY{n}{df}\PY{p}{[}\PY{n}{cols}\PY{p}{]}\PY{o}{.}\PY{n}{var}\PY{p}{(}\PY{p}{)}\PY{p}{)}\PY{p}{,} \PY{n}{columns}\PY{o}{=}\PY{n}{cols}\PY{p}{,} \PY{n}{index}\PY{o}{=}\PY{n}{cols}\PY{p}{)}
        
            \PY{c+c1}{\PYZsh{} populate cov\PYZus{}df empty dataframe with covariances}
            \PY{k}{for} \PY{n}{X} \PY{o+ow}{in} \PY{n}{cols}\PY{p}{:}
                \PY{k}{for} \PY{n}{Y} \PY{o+ow}{in} \PY{n}{cols}\PY{p}{:}
                    \PY{c+c1}{\PYZsh{} we already have the variances, so ignore when X == Y}
                    \PY{k}{if} \PY{n}{X} \PY{o}{!=} \PY{n}{Y}\PY{p}{:}
                        \PY{c+c1}{\PYZsh{} populate empty dataframe with symmetrical value}
                        \PY{c+c1}{\PYZsh{} that is off the diagonal}
                        \PY{n}{\PYZus{}cov} \PY{o}{=} \PY{n}{cov\PYZus{}xy}\PY{p}{(}\PY{n}{df}\PY{p}{[}\PY{n}{X}\PY{p}{]}\PY{p}{,} \PY{n}{df}\PY{p}{[}\PY{n}{Y}\PY{p}{]}\PY{p}{)}\PY{p}{[}\PY{l+m+mi}{0}\PY{p}{,} \PY{l+m+mi}{1}\PY{p}{]}
                        \PY{n}{cov\PYZus{}df}\PY{o}{.}\PY{n}{loc}\PY{p}{[}\PY{n}{Y}\PY{p}{,} \PY{n}{X}\PY{p}{]} \PY{o}{=}  \PY{n}{\PYZus{}cov}
                    \PY{k}{else}\PY{p}{:}
                        \PY{k}{pass}
            \PY{k}{return} \PY{n}{cov\PYZus{}df}
        
        \PY{n}{cols} \PY{o}{=} \PY{p}{[}\PY{l+s+s1}{\PYZsq{}}\PY{l+s+s1}{mcg}\PY{l+s+s1}{\PYZsq{}}\PY{p}{,} \PY{l+s+s1}{\PYZsq{}}\PY{l+s+s1}{gvh}\PY{l+s+s1}{\PYZsq{}}\PY{p}{,} \PY{l+s+s1}{\PYZsq{}}\PY{l+s+s1}{lip}\PY{l+s+s1}{\PYZsq{}}\PY{p}{,} \PY{l+s+s1}{\PYZsq{}}\PY{l+s+s1}{chg}\PY{l+s+s1}{\PYZsq{}}\PY{p}{,} \PY{l+s+s1}{\PYZsq{}}\PY{l+s+s1}{aac}\PY{l+s+s1}{\PYZsq{}}\PY{p}{,} \PY{l+s+s1}{\PYZsq{}}\PY{l+s+s1}{alm1}\PY{l+s+s1}{\PYZsq{}}\PY{p}{,} \PY{l+s+s1}{\PYZsq{}}\PY{l+s+s1}{alm2}\PY{l+s+s1}{\PYZsq{}}\PY{p}{]}
        \PY{n}{C} \PY{o}{=} \PY{n}{cov\PYZus{}matrix}\PY{p}{(}\PY{n}{df}\PY{p}{,} \PY{n}{cols}\PY{p}{)}
        \PY{n}{C}
\end{Verbatim}


\begin{Verbatim}[commandchars=\\\{\}]
{\color{outcolor}Out[{\color{outcolor}6}]:}            mcg       gvh       lip       chg       aac      alm1      alm2
        mcg   0.037882  0.013115  0.002529  0.000373  0.005257  0.016670  0.006810
        gvh   0.013115  0.021950  0.000574  0.000075  0.001266  0.005546 -0.003729
        lip   0.002529  0.000574  0.007831  0.000753  0.000760  0.001829 -0.001067
        chg   0.000373  0.000075  0.000753  0.000744 -0.000149 -0.000045 -0.000298
        aac   0.005257  0.001266  0.000760 -0.000149  0.014976  0.007379  0.006475
        alm1  0.016670  0.005546  0.001829 -0.000045  0.007379  0.046549  0.036566
        alm2  0.006810 -0.003729 -0.001067 -0.000298  0.006475  0.036566  0.043853
\end{Verbatim}
            
    \begin{Verbatim}[commandchars=\\\{\}]
{\color{incolor}In [{\color{incolor}7}]:} \PY{c+c1}{\PYZsh{} make sure above is correct}
        \PY{n}{pd}\PY{o}{.}\PY{n}{DataFrame}\PY{p}{(}\PY{n}{np}\PY{o}{.}\PY{n}{cov}\PY{p}{(}\PY{n}{df}\PY{p}{[}\PY{n}{cols}\PY{p}{]}\PY{o}{.}\PY{n}{T}\PY{p}{)}\PY{p}{,} \PY{n}{columns}\PY{o}{=}\PY{n}{cols}\PY{p}{,} \PY{n}{index}\PY{o}{=}\PY{n}{cols}\PY{p}{)}
\end{Verbatim}


\begin{Verbatim}[commandchars=\\\{\}]
{\color{outcolor}Out[{\color{outcolor}7}]:}            mcg       gvh       lip       chg       aac      alm1      alm2
        mcg   0.037882  0.013115  0.002529  0.000373  0.005257  0.016670  0.006810
        gvh   0.013115  0.021950  0.000574  0.000075  0.001266  0.005546 -0.003729
        lip   0.002529  0.000574  0.007831  0.000753  0.000760  0.001829 -0.001067
        chg   0.000373  0.000075  0.000753  0.000744 -0.000149 -0.000045 -0.000298
        aac   0.005257  0.001266  0.000760 -0.000149  0.014976  0.007379  0.006475
        alm1  0.016670  0.005546  0.001829 -0.000045  0.007379  0.046549  0.036566
        alm2  0.006810 -0.003729 -0.001067 -0.000298  0.006475  0.036566  0.043853
\end{Verbatim}
            
    \begin{Verbatim}[commandchars=\\\{\}]
{\color{incolor}In [{\color{incolor}8}]:} \PY{c+c1}{\PYZsh{} double\PYZhy{}check to make sure above is correct}
        \PY{n+nb}{print}\PY{p}{(}\PY{n}{np}\PY{o}{.}\PY{n}{allclose}\PY{p}{(}\PY{n}{cov\PYZus{}matrix}\PY{p}{(}\PY{n}{df}\PY{p}{,} \PY{n}{cols}\PY{p}{)}\PY{p}{,} \PY{n}{df}\PY{p}{[}\PY{n}{cols}\PY{p}{]}\PY{o}{.}\PY{n}{cov}\PY{p}{(}\PY{p}{)}\PY{p}{)}\PY{p}{)}
        \PY{n+nb}{print}\PY{p}{(}\PY{n}{np}\PY{o}{.}\PY{n}{allclose}\PY{p}{(}\PY{n}{cov\PYZus{}matrix}\PY{p}{(}\PY{n}{df}\PY{p}{,} \PY{n}{cols}\PY{p}{)}\PY{p}{,} \PY{n}{np}\PY{o}{.}\PY{n}{cov}\PY{p}{(}\PY{n}{df}\PY{p}{[}\PY{n}{cols}\PY{p}{]}\PY{o}{.}\PY{n}{T}\PY{p}{)}\PY{p}{)}\PY{p}{)}
\end{Verbatim}


    \begin{Verbatim}[commandchars=\\\{\}]
True
True

    \end{Verbatim}

    \subsubsection{Problem 2c (10 points).}\label{problem-2c-10-points.}

Compute the decomposition of the covariance matrix using singular value
decomposition. Using a python function is definitely the way to go here.

    \begin{Verbatim}[commandchars=\\\{\}]
{\color{incolor}In [{\color{incolor}9}]:} \PY{n}{u}\PY{p}{,} \PY{n}{s}\PY{p}{,} \PY{n}{v} \PY{o}{=} \PY{n}{np}\PY{o}{.}\PY{n}{linalg}\PY{o}{.}\PY{n}{svd}\PY{p}{(}\PY{n}{C}\PY{p}{)}
        \PY{n+nb}{print}\PY{p}{(}\PY{l+s+s1}{\PYZsq{}}\PY{l+s+s1}{u = }\PY{l+s+s1}{\PYZsq{}}\PY{p}{,} \PY{n}{u}\PY{p}{,} \PY{l+s+s1}{\PYZsq{}}\PY{l+s+se}{\PYZbs{}n}\PY{l+s+s1}{\PYZsq{}}\PY{p}{)}
        \PY{n+nb}{print}\PY{p}{(}\PY{l+s+s1}{\PYZsq{}}\PY{l+s+s1}{s = }\PY{l+s+s1}{\PYZsq{}}\PY{p}{,} \PY{n}{s}\PY{p}{,} \PY{l+s+s1}{\PYZsq{}}\PY{l+s+se}{\PYZbs{}n}\PY{l+s+s1}{\PYZsq{}}\PY{p}{)}
        \PY{n+nb}{print}\PY{p}{(}\PY{l+s+s1}{\PYZsq{}}\PY{l+s+s1}{v = }\PY{l+s+s1}{\PYZsq{}}\PY{p}{,} \PY{n}{v}\PY{p}{,} \PY{l+s+s1}{\PYZsq{}}\PY{l+s+se}{\PYZbs{}n}\PY{l+s+s1}{\PYZsq{}}\PY{p}{)}
        \PY{n+nb}{print}\PY{p}{(}\PY{l+s+s1}{\PYZsq{}}\PY{l+s+s1}{Note same as covariance matrix:}\PY{l+s+s1}{\PYZsq{}}\PY{p}{)}
        \PY{n}{pd}\PY{o}{.}\PY{n}{DataFrame}\PY{p}{(}\PY{n}{np}\PY{o}{.}\PY{n}{dot}\PY{p}{(}\PY{n}{u}\PY{p}{,} \PY{n}{np}\PY{o}{.}\PY{n}{dot}\PY{p}{(}\PY{n}{np}\PY{o}{.}\PY{n}{diag}\PY{p}{(}\PY{n}{s}\PY{p}{)}\PY{p}{,} \PY{n}{v}\PY{p}{)}\PY{p}{)}\PY{p}{)}
\end{Verbatim}


    \begin{Verbatim}[commandchars=\\\{\}]
u =  [[-3.41720629e-01 -7.29824958e-01  4.56914809e-01  3.52555665e-01
  -1.29441525e-01  2.62199804e-02 -8.56193962e-03]
 [-9.17492644e-02 -5.28794379e-01 -7.04480258e-01 -3.42514767e-01
  -1.39018617e-01  2.81111220e-01  2.89703065e-04]
 [-1.99698823e-02 -7.25012267e-02  9.24097367e-02 -1.46705018e-02
   8.54762221e-01  4.93627297e-01 -1.06318528e-01]
 [ 9.74209562e-04 -1.16809378e-02  8.81896510e-03  1.64849996e-02
   8.61437517e-02  6.21081678e-02  9.94100049e-01]
 [-1.47115119e-01 -4.80496808e-02  4.70463079e-01 -8.66343630e-01
  -4.86492931e-02 -3.90505852e-02  1.64278490e-02]
 [-6.89914858e-01  7.21275410e-02 -2.54072739e-01 -1.83906883e-02
   3.50569568e-01 -5.75266798e-01  9.64476070e-03]
 [-6.13827312e-01  4.18124394e-01  2.07514962e-02  8.37356323e-02
  -3.17196689e-01  5.83358230e-01 -5.01770938e-03]] 

s =  [0.08970253 0.04243901 0.01463252 0.01288363 0.00853362 0.00493527
 0.00065892] 

v =  [[-3.41720629e-01 -9.17492644e-02 -1.99698823e-02  9.74209562e-04
  -1.47115119e-01 -6.89914858e-01 -6.13827312e-01]
 [-7.29824958e-01 -5.28794379e-01 -7.25012267e-02 -1.16809378e-02
  -4.80496808e-02  7.21275410e-02  4.18124394e-01]
 [ 4.56914809e-01 -7.04480258e-01  9.24097367e-02  8.81896510e-03
   4.70463079e-01 -2.54072739e-01  2.07514962e-02]
 [ 3.52555665e-01 -3.42514767e-01 -1.46705018e-02  1.64849996e-02
  -8.66343630e-01 -1.83906883e-02  8.37356323e-02]
 [-1.29441525e-01 -1.39018617e-01  8.54762221e-01  8.61437517e-02
  -4.86492931e-02  3.50569568e-01 -3.17196689e-01]
 [ 2.62199804e-02  2.81111220e-01  4.93627297e-01  6.21081678e-02
  -3.90505852e-02 -5.75266798e-01  5.83358230e-01]
 [-8.56193962e-03  2.89703065e-04 -1.06318528e-01  9.94100049e-01
   1.64278490e-02  9.64476070e-03 -5.01770938e-03]] 

Note same as covariance matrix:

    \end{Verbatim}

\begin{Verbatim}[commandchars=\\\{\}]
{\color{outcolor}Out[{\color{outcolor}9}]:}           0         1         2         3         4         5         6
        0  0.037882  0.013115  0.002529  0.000373  0.005257  0.016670  0.006810
        1  0.013115  0.021950  0.000574  0.000075  0.001266  0.005546 -0.003729
        2  0.002529  0.000574  0.007831  0.000753  0.000760  0.001829 -0.001067
        3  0.000373  0.000075  0.000753  0.000744 -0.000149 -0.000045 -0.000298
        4  0.005257  0.001266  0.000760 -0.000149  0.014976  0.007379  0.006475
        5  0.016670  0.005546  0.001829 -0.000045  0.007379  0.046549  0.036566
        6  0.006810 -0.003729 -0.001067 -0.000298  0.006475  0.036566  0.043853
\end{Verbatim}
            
    \subsubsection{Problem 2d (10 points)}\label{problem-2d-10-points}

Compute the projection of the raw data onto the appropriate two
eigenvectors. Consider which columns should be projected and the
normalizations.

    \begin{Verbatim}[commandchars=\\\{\}]
{\color{incolor}In [{\color{incolor}10}]:} \PY{c+c1}{\PYZsh{} subtract mean}
         \PY{n}{cols} \PY{o}{=} \PY{p}{[}\PY{l+s+s1}{\PYZsq{}}\PY{l+s+s1}{mcg}\PY{l+s+s1}{\PYZsq{}}\PY{p}{,} \PY{l+s+s1}{\PYZsq{}}\PY{l+s+s1}{gvh}\PY{l+s+s1}{\PYZsq{}}\PY{p}{,} \PY{l+s+s1}{\PYZsq{}}\PY{l+s+s1}{lip}\PY{l+s+s1}{\PYZsq{}}\PY{p}{,} \PY{l+s+s1}{\PYZsq{}}\PY{l+s+s1}{chg}\PY{l+s+s1}{\PYZsq{}}\PY{p}{,} \PY{l+s+s1}{\PYZsq{}}\PY{l+s+s1}{aac}\PY{l+s+s1}{\PYZsq{}}\PY{p}{,} \PY{l+s+s1}{\PYZsq{}}\PY{l+s+s1}{alm1}\PY{l+s+s1}{\PYZsq{}}\PY{p}{,} \PY{l+s+s1}{\PYZsq{}}\PY{l+s+s1}{alm2}\PY{l+s+s1}{\PYZsq{}}\PY{p}{]}
         \PY{n}{df\PYZus{}0} \PY{o}{=} \PY{n}{df}\PY{p}{[}\PY{n}{cols}\PY{p}{]} \PY{o}{\PYZhy{}} \PY{n}{df}\PY{p}{[}\PY{n}{cols}\PY{p}{]}\PY{o}{.}\PY{n}{mean}\PY{p}{(}\PY{n}{axis}\PY{o}{=}\PY{l+m+mi}{0}\PY{p}{)}
         
         \PY{c+c1}{\PYZsh{} get covariance matrix}
         \PY{n}{C} \PY{o}{=} \PY{n}{cov\PYZus{}matrix}\PY{p}{(}\PY{n}{df\PYZus{}0}\PY{p}{,} \PY{n}{cols}\PY{p}{)}
         
         \PY{k}{assert} \PY{n}{np}\PY{o}{.}\PY{n}{linalg}\PY{o}{.}\PY{n}{det}\PY{p}{(}\PY{n}{C}\PY{p}{)} \PY{o}{\PYZgt{}}\PY{o}{=} \PY{l+m+mi}{0} 
         
         \PY{n}{eigenvalues}\PY{p}{,} \PY{n}{eigenvectors} \PY{o}{=} \PY{n}{np}\PY{o}{.}\PY{n}{linalg}\PY{o}{.}\PY{n}{eigh}\PY{p}{(}\PY{n}{C}\PY{p}{)}
         \PY{n+nb}{print}\PY{p}{(}\PY{l+s+s1}{\PYZsq{}}\PY{l+s+s1}{Eigenvalues }\PY{l+s+se}{\PYZbs{}n}\PY{l+s+si}{\PYZpc{}s}\PY{l+s+se}{\PYZbs{}n}\PY{l+s+s1}{\PYZsq{}} \PY{o}{\PYZpc{}}\PY{k}{eigenvalues})
         \PY{n+nb}{print}\PY{p}{(}\PY{l+s+s1}{\PYZsq{}}\PY{l+s+s1}{Eigenvectors }\PY{l+s+se}{\PYZbs{}n}\PY{l+s+si}{\PYZpc{}s}\PY{l+s+se}{\PYZbs{}n}\PY{l+s+s1}{\PYZsq{}} \PY{o}{\PYZpc{}}\PY{k}{eigenvectors})
         
         \PY{k}{for} \PY{n}{vec} \PY{o+ow}{in} \PY{n}{eigenvectors}\PY{p}{:}
             \PY{k}{assert} \PY{n}{np}\PY{o}{.}\PY{n}{allclose}\PY{p}{(}\PY{l+m+mi}{1}\PY{p}{,} \PY{n}{np}\PY{o}{.}\PY{n}{linalg}\PY{o}{.}\PY{n}{norm}\PY{p}{(}\PY{n}{vec}\PY{p}{)}\PY{p}{)}
         \PY{c+c1}{\PYZsh{}print(eigenvectors[:,1])}
         
         \PY{c+c1}{\PYZsh{} sort eigenvalues in desc order}
         \PY{n}{idx} \PY{o}{=} \PY{n}{np}\PY{o}{.}\PY{n}{argsort} \PY{p}{(}\PY{o}{\PYZhy{}}\PY{n}{eigenvalues}\PY{p}{)}
         \PY{n}{eigenvalues} \PY{o}{=} \PY{n}{eigenvalues} \PY{p}{[} \PY{n}{idx} \PY{p}{]}
         \PY{n}{eigenvectors} \PY{o}{=} \PY{n}{eigenvectors} \PY{p}{[}\PY{p}{:} \PY{p}{,} \PY{n}{idx} \PY{p}{]}
         
         \PY{n+nb}{print}\PY{p}{(}\PY{l+s+s1}{\PYZsq{}}\PY{l+s+s1}{eigenvalues = }\PY{l+s+se}{\PYZbs{}n}\PY{l+s+s1}{\PYZsq{}}\PY{p}{,} \PY{n}{eigenvalues}\PY{p}{,} \PY{l+s+s1}{\PYZsq{}}\PY{l+s+se}{\PYZbs{}n}\PY{l+s+s1}{\PYZsq{}}\PY{p}{)}
         \PY{n+nb}{print}\PY{p}{(}\PY{l+s+s1}{\PYZsq{}}\PY{l+s+s1}{sorted idx = }\PY{l+s+s1}{\PYZsq{}}\PY{p}{,} \PY{n}{idx}\PY{p}{,} \PY{l+s+s1}{\PYZsq{}}\PY{l+s+se}{\PYZbs{}n}\PY{l+s+s1}{\PYZsq{}}\PY{p}{)}
         \PY{n+nb}{print}\PY{p}{(}\PY{l+s+s1}{\PYZsq{}}\PY{l+s+s1}{sorted eigenvectors = }\PY{l+s+s1}{\PYZsq{}}\PY{p}{,} \PY{n}{eigenvectors}\PY{p}{,} \PY{l+s+s1}{\PYZsq{}}\PY{l+s+se}{\PYZbs{}n}\PY{l+s+s1}{\PYZsq{}}\PY{p}{)}
         
         \PY{n+nb}{print}\PY{p}{(}\PY{l+s+s1}{\PYZsq{}}\PY{l+s+s1}{eigenvectors[:,0] = }\PY{l+s+s1}{\PYZsq{}}\PY{p}{,} \PY{n}{eigenvectors}\PY{p}{[}\PY{p}{:}\PY{p}{,}\PY{l+m+mi}{0}\PY{p}{]}\PY{p}{)}
         \PY{n+nb}{print}\PY{p}{(}\PY{l+s+s1}{\PYZsq{}}\PY{l+s+s1}{eigenvectors[:,1] = }\PY{l+s+s1}{\PYZsq{}}\PY{p}{,} \PY{n}{eigenvectors}\PY{p}{[}\PY{p}{:}\PY{p}{,}\PY{l+m+mi}{1}\PY{p}{]}\PY{p}{)}
\end{Verbatim}


    \begin{Verbatim}[commandchars=\\\{\}]
Eigenvalues 
[0.00065892 0.00493527 0.00853362 0.01288363 0.01463252 0.04243901
 0.08970253]

Eigenvectors 
[[-8.56193962e-03  2.62199804e-02 -1.29441525e-01  3.52555665e-01
   4.56914809e-01 -7.29824958e-01 -3.41720629e-01]
 [ 2.89703065e-04  2.81111220e-01 -1.39018617e-01 -3.42514767e-01
  -7.04480258e-01 -5.28794379e-01 -9.17492644e-02]
 [-1.06318528e-01  4.93627297e-01  8.54762221e-01 -1.46705018e-02
   9.24097367e-02 -7.25012267e-02 -1.99698823e-02]
 [ 9.94100049e-01  6.21081678e-02  8.61437517e-02  1.64849996e-02
   8.81896510e-03 -1.16809378e-02  9.74209562e-04]
 [ 1.64278490e-02 -3.90505852e-02 -4.86492931e-02 -8.66343630e-01
   4.70463079e-01 -4.80496808e-02 -1.47115119e-01]
 [ 9.64476070e-03 -5.75266798e-01  3.50569568e-01 -1.83906883e-02
  -2.54072739e-01  7.21275410e-02 -6.89914858e-01]
 [-5.01770938e-03  5.83358230e-01 -3.17196689e-01  8.37356323e-02
   2.07514962e-02  4.18124394e-01 -6.13827312e-01]]

eigenvalues = 
 [0.08970253 0.04243901 0.01463252 0.01288363 0.00853362 0.00493527
 0.00065892] 

sorted idx =  [6 5 4 3 2 1 0] 

sorted eigenvectors =  [[-3.41720629e-01 -7.29824958e-01  4.56914809e-01  3.52555665e-01
  -1.29441525e-01  2.62199804e-02 -8.56193962e-03]
 [-9.17492644e-02 -5.28794379e-01 -7.04480258e-01 -3.42514767e-01
  -1.39018617e-01  2.81111220e-01  2.89703065e-04]
 [-1.99698823e-02 -7.25012267e-02  9.24097367e-02 -1.46705018e-02
   8.54762221e-01  4.93627297e-01 -1.06318528e-01]
 [ 9.74209562e-04 -1.16809378e-02  8.81896510e-03  1.64849996e-02
   8.61437517e-02  6.21081678e-02  9.94100049e-01]
 [-1.47115119e-01 -4.80496808e-02  4.70463079e-01 -8.66343630e-01
  -4.86492931e-02 -3.90505852e-02  1.64278490e-02]
 [-6.89914858e-01  7.21275410e-02 -2.54072739e-01 -1.83906883e-02
   3.50569568e-01 -5.75266798e-01  9.64476070e-03]
 [-6.13827312e-01  4.18124394e-01  2.07514962e-02  8.37356323e-02
  -3.17196689e-01  5.83358230e-01 -5.01770938e-03]] 

eigenvectors[:,0] =  [-0.34172063 -0.09174926 -0.01996988  0.00097421 -0.14711512 -0.68991486
 -0.61382731]
eigenvectors[:,1] =  [-0.72982496 -0.52879438 -0.07250123 -0.01168094 -0.04804968  0.07212754
  0.41812439]

    \end{Verbatim}

    \begin{Verbatim}[commandchars=\\\{\}]
{\color{incolor}In [{\color{incolor}11}]:} \PY{c+c1}{\PYZsh{} get principal component projections}
         \PY{n}{df\PYZus{}0}\PY{p}{[}\PY{n}{cols}\PY{p}{]}\PY{o}{.}\PY{n}{dot}\PY{p}{(}\PY{n}{eigenvectors}\PY{p}{)}\PY{p}{[}\PY{p}{[}\PY{l+m+mi}{0}\PY{p}{,} \PY{l+m+mi}{1}\PY{p}{]}\PY{p}{]}\PY{o}{.}\PY{n}{T}
\end{Verbatim}


\begin{Verbatim}[commandchars=\\\{\}]
{\color{outcolor}Out[{\color{outcolor}11}]:}         0         1         2         3         4         5         6    \textbackslash{}
         0  0.285601  0.290838  0.104676  0.087943  0.366268  0.080230  0.385554   
         1  0.035274  0.330159 -0.015248 -0.122218  0.210366 -0.083273  0.187390   
         
                 7         8         9      {\ldots}          326       327       328  \textbackslash{}
         0  0.331901  0.064330  0.371515    {\ldots}     0.092863  0.198870  0.087381   
         1  0.235198  0.283871  0.002960    {\ldots}    -0.310786 -0.153296 -0.322197   
         
                 329       330       331       332       333       334       335  
         0 -0.028480 -0.016919 -0.084233  0.139026  0.111904  0.106204 -0.108764  
         1 -0.262023  0.028185 -0.274803 -0.274116 -0.187103 -0.178851 -0.281129  
         
         [2 rows x 336 columns]
\end{Verbatim}
            
    \begin{Verbatim}[commandchars=\\\{\}]
{\color{incolor}In [{\color{incolor}12}]:} \PY{c+c1}{\PYZsh{} check to make sure PCA results return same\PYZhy{}ish as above}
         \PY{n}{pd}\PY{o}{.}\PY{n}{DataFrame}\PY{p}{(}\PY{n}{PCA}\PY{p}{(}\PY{n}{n\PYZus{}components}\PY{o}{=}\PY{l+m+mi}{2}\PY{p}{)}\PY{o}{.}\PY{n}{fit\PYZus{}transform}\PY{p}{(}\PY{n}{df}\PY{p}{[}\PY{n}{cols}\PY{p}{]}\PY{p}{)}\PY{p}{)}\PY{o}{.}\PY{n}{T}
\end{Verbatim}


\begin{Verbatim}[commandchars=\\\{\}]
{\color{outcolor}Out[{\color{outcolor}12}]:}         0         1         2         3         4         5         6    \textbackslash{}
         0 -0.285601 -0.290838 -0.104676 -0.087943 -0.366268 -0.080230 -0.385554   
         1 -0.035274 -0.330159  0.015248  0.122218 -0.210366  0.083273 -0.187390   
         
                 7         8         9      {\ldots}          326       327       328  \textbackslash{}
         0 -0.331901 -0.064330 -0.371515    {\ldots}    -0.092863 -0.198870 -0.087381   
         1 -0.235198 -0.283871 -0.002960    {\ldots}     0.310786  0.153296  0.322197   
         
                 329       330       331       332       333       334       335  
         0  0.028480  0.016919  0.084233 -0.139026 -0.111904 -0.106204  0.108764  
         1  0.262023 -0.028185  0.274803  0.274116  0.187103  0.178851  0.281129  
         
         [2 rows x 336 columns]
\end{Verbatim}
            
    \subsubsection{Problem 2e (10 points)}\label{problem-2e-10-points}

Plot the projected points such that the 8 different classes can be
visually identified. Be sure to label the classes and axes. Commont on
the quality of the separation of the different classes using PCA.

    \begin{Verbatim}[commandchars=\\\{\}]
{\color{incolor}In [{\color{incolor}13}]:} \PY{k}{def} \PY{n+nf}{random\PYZus{}hex}\PY{p}{(}\PY{n}{seed}\PY{p}{)}\PY{p}{:}
             \PY{n}{np}\PY{o}{.}\PY{n}{random}\PY{o}{.}\PY{n}{seed}\PY{p}{(}\PY{n}{seed}\PY{p}{)}
             \PY{n}{r} \PY{o}{=} \PY{k}{lambda}\PY{p}{:} \PY{n}{random}\PY{o}{.}\PY{n}{randint}\PY{p}{(}\PY{l+m+mi}{0}\PY{p}{,} \PY{l+m+mi}{255}\PY{p}{)}
             \PY{k}{return} \PY{l+s+s1}{\PYZsq{}}\PY{l+s+s1}{\PYZsh{}}\PY{l+s+si}{\PYZpc{}02X}\PY{l+s+si}{\PYZpc{}02X}\PY{l+s+si}{\PYZpc{}02X}\PY{l+s+s1}{\PYZsq{}} \PY{o}{\PYZpc{}} \PY{p}{(}\PY{n}{r}\PY{p}{(}\PY{p}{)}\PY{p}{,}\PY{n}{r}\PY{p}{(}\PY{p}{)}\PY{p}{,}\PY{n}{r}\PY{p}{(}\PY{p}{)}\PY{p}{)}
         
         \PY{c+c1}{\PYZsh{} map targets to colors randomly}
         \PY{n}{unq\PYZus{}tgts} \PY{o}{=} \PY{n}{df}\PY{o}{.}\PY{n}{target}\PY{o}{.}\PY{n}{unique}\PY{p}{(}\PY{p}{)}
         \PY{n}{colors} \PY{o}{=} \PY{p}{[}\PY{n}{random\PYZus{}hex}\PY{p}{(}\PY{n}{i}\PY{p}{)} \PY{k}{for} \PY{n}{i}\PY{p}{,}\PY{n+nb+bp}{cls} \PY{o+ow}{in} \PY{n+nb}{zip}\PY{p}{(}\PY{n+nb}{range}\PY{p}{(}\PY{n+nb}{len}\PY{p}{(}\PY{n}{unq\PYZus{}tgts}\PY{p}{)}\PY{p}{)}\PY{p}{,} \PY{n}{unq\PYZus{}tgts}\PY{p}{)}\PY{p}{]}
         
         \PY{c+c1}{\PYZsh{} do projection by hand}
         \PY{n}{df\PYZus{}pca} \PY{o}{=} \PY{n}{df\PYZus{}0}\PY{p}{[}\PY{n}{cols}\PY{p}{]}\PY{o}{.}\PY{n}{dot}\PY{p}{(}\PY{n}{eigenvectors}\PY{p}{)}\PY{p}{[}\PY{p}{[}\PY{l+m+mi}{0}\PY{p}{,} \PY{l+m+mi}{1}\PY{p}{]}\PY{p}{]}
         \PY{n}{df\PYZus{}pca}\PY{p}{[}\PY{l+s+s1}{\PYZsq{}}\PY{l+s+s1}{target}\PY{l+s+s1}{\PYZsq{}}\PY{p}{]} \PY{o}{=} \PY{n}{df}\PY{p}{[}\PY{l+s+s1}{\PYZsq{}}\PY{l+s+s1}{target}\PY{l+s+s1}{\PYZsq{}}\PY{p}{]}
         
         \PY{c+c1}{\PYZsh{} plot the first two PCs}
         \PY{n}{fig}\PY{p}{,} \PY{n}{axes} \PY{o}{=} \PY{n}{plt}\PY{o}{.}\PY{n}{subplots}\PY{p}{(}\PY{l+m+mi}{1}\PY{p}{,} \PY{l+m+mi}{2}\PY{p}{,} \PY{n}{figsize}\PY{o}{=}\PY{p}{(}\PY{l+m+mi}{17}\PY{p}{,} \PY{l+m+mi}{7}\PY{p}{)}\PY{p}{)}
         
         \PY{n}{ax} \PY{o}{=} \PY{n}{axes}\PY{p}{[}\PY{l+m+mi}{0}\PY{p}{]}
         \PY{n}{i} \PY{o}{=} \PY{l+m+mi}{0}
         \PY{k}{for} \PY{n}{target}\PY{p}{,} \PY{n}{\PYZus{}df} \PY{o+ow}{in} \PY{n}{df\PYZus{}pca}\PY{o}{.}\PY{n}{groupby}\PY{p}{(}\PY{l+s+s1}{\PYZsq{}}\PY{l+s+s1}{target}\PY{l+s+s1}{\PYZsq{}}\PY{p}{)}\PY{p}{:}
             \PY{n}{color} \PY{o}{=} \PY{n}{colors}\PY{p}{[}\PY{n}{i}\PY{p}{]}
             \PY{n}{label} \PY{o}{=} \PY{n}{unq\PYZus{}tgts}\PY{p}{[}\PY{n}{i}\PY{p}{]}
             \PY{n}{ax}\PY{o}{.}\PY{n}{scatter}\PY{p}{(}\PY{n}{\PYZus{}df}\PY{p}{[}\PY{l+m+mi}{0}\PY{p}{]}\PY{p}{,} \PY{n}{\PYZus{}df}\PY{p}{[}\PY{l+m+mi}{1}\PY{p}{]}\PY{p}{,} \PY{n}{color}\PY{o}{=}\PY{n}{color}\PY{p}{,} \PY{n}{label}\PY{o}{=}\PY{n}{label}\PY{p}{,} \PY{n}{alpha}\PY{o}{=}\PY{o}{.}\PY{l+m+mi}{65}\PY{p}{)}
             \PY{n}{i} \PY{o}{+}\PY{o}{=} \PY{l+m+mi}{1}
         \PY{n}{ax}\PY{o}{.}\PY{n}{grid}\PY{p}{(}\PY{n}{alpha}\PY{o}{=}\PY{o}{.}\PY{l+m+mi}{7}\PY{p}{)}
         \PY{n}{sns}\PY{o}{.}\PY{n}{despine}\PY{p}{(}\PY{p}{)}
         \PY{n}{ax}\PY{o}{.}\PY{n}{set\PYZus{}xlabel}\PY{p}{(}\PY{l+s+s1}{\PYZsq{}}\PY{l+s+s1}{1st Principal Component}\PY{l+s+s1}{\PYZsq{}}\PY{p}{)}
         \PY{n}{ax}\PY{o}{.}\PY{n}{set\PYZus{}ylabel}\PY{p}{(}\PY{l+s+s1}{\PYZsq{}}\PY{l+s+s1}{2nd Principal Component}\PY{l+s+s1}{\PYZsq{}}\PY{p}{)}
         \PY{n}{ax}\PY{o}{.}\PY{n}{legend}\PY{p}{(}\PY{n}{loc}\PY{o}{=}\PY{l+s+s1}{\PYZsq{}}\PY{l+s+s1}{best}\PY{l+s+s1}{\PYZsq{}}\PY{p}{)}
         \PY{n}{ax}\PY{o}{.}\PY{n}{set\PYZus{}title}\PY{p}{(}\PY{l+s+s1}{\PYZsq{}}\PY{l+s+s1}{Hand Computed: Top 2 Principal Components}\PY{l+s+se}{\PYZbs{}n}\PY{l+s+s1}{Colored by Class}\PY{l+s+s1}{\PYZsq{}}\PY{p}{)}
         
         \PY{c+c1}{\PYZsh{} perform PCA using sklearn}
         \PY{n}{pca} \PY{o}{=} \PY{n}{PCA}\PY{p}{(}\PY{n}{n\PYZus{}components}\PY{o}{=}\PY{l+m+mi}{2}\PY{p}{,} \PY{n}{random\PYZus{}state}\PY{o}{=}\PY{l+m+mi}{77}\PY{p}{)}
         \PY{n}{pca}\PY{o}{.}\PY{n}{fit}\PY{p}{(}\PY{n}{df}\PY{p}{[}\PY{n}{cols}\PY{p}{]}\PY{p}{)}
         \PY{n}{df\PYZus{}pca} \PY{o}{=} \PY{n}{pd}\PY{o}{.}\PY{n}{DataFrame}\PY{p}{(}\PY{n}{pca}\PY{o}{.}\PY{n}{transform}\PY{p}{(}\PY{n}{df}\PY{p}{[}\PY{n}{cols}\PY{p}{]}\PY{p}{)}\PY{p}{)}
         
         \PY{c+c1}{\PYZsh{} add target for plotting}
         \PY{n}{df\PYZus{}pca}\PY{p}{[}\PY{l+s+s1}{\PYZsq{}}\PY{l+s+s1}{target}\PY{l+s+s1}{\PYZsq{}}\PY{p}{]} \PY{o}{=} \PY{n}{df}\PY{p}{[}\PY{l+s+s1}{\PYZsq{}}\PY{l+s+s1}{target}\PY{l+s+s1}{\PYZsq{}}\PY{p}{]}
         
         \PY{c+c1}{\PYZsh{} plot the first two PCs}
         \PY{n}{ax} \PY{o}{=} \PY{n}{axes}\PY{p}{[}\PY{l+m+mi}{1}\PY{p}{]}
         \PY{n}{i} \PY{o}{=} \PY{l+m+mi}{0}
         \PY{k}{for} \PY{n}{target}\PY{p}{,} \PY{n}{\PYZus{}df} \PY{o+ow}{in} \PY{n}{df\PYZus{}pca}\PY{o}{.}\PY{n}{groupby}\PY{p}{(}\PY{l+s+s1}{\PYZsq{}}\PY{l+s+s1}{target}\PY{l+s+s1}{\PYZsq{}}\PY{p}{)}\PY{p}{:}
             \PY{n}{color} \PY{o}{=} \PY{n}{colors}\PY{p}{[}\PY{n}{i}\PY{p}{]}
             \PY{n}{label} \PY{o}{=} \PY{n}{unq\PYZus{}tgts}\PY{p}{[}\PY{n}{i}\PY{p}{]}
             \PY{n}{ax}\PY{o}{.}\PY{n}{scatter}\PY{p}{(}\PY{n}{\PYZus{}df}\PY{p}{[}\PY{l+m+mi}{0}\PY{p}{]}\PY{p}{,} \PY{n}{\PYZus{}df}\PY{p}{[}\PY{l+m+mi}{1}\PY{p}{]}\PY{p}{,} \PY{n}{color}\PY{o}{=}\PY{n}{color}\PY{p}{,} \PY{n}{label}\PY{o}{=}\PY{n}{label}\PY{p}{,} \PY{n}{alpha}\PY{o}{=}\PY{o}{.}\PY{l+m+mi}{65}\PY{p}{)}
             \PY{n}{i} \PY{o}{+}\PY{o}{=} \PY{l+m+mi}{1}
         \PY{n}{ax}\PY{o}{.}\PY{n}{grid}\PY{p}{(}\PY{n}{alpha}\PY{o}{=}\PY{o}{.}\PY{l+m+mi}{7}\PY{p}{)}
         \PY{n}{sns}\PY{o}{.}\PY{n}{despine}\PY{p}{(}\PY{p}{)}
         \PY{n}{ax}\PY{o}{.}\PY{n}{set\PYZus{}xlabel}\PY{p}{(}\PY{l+s+s1}{\PYZsq{}}\PY{l+s+s1}{1st Principal Component}\PY{l+s+s1}{\PYZsq{}}\PY{p}{)}
         \PY{n}{ax}\PY{o}{.}\PY{n}{set\PYZus{}ylabel}\PY{p}{(}\PY{l+s+s1}{\PYZsq{}}\PY{l+s+s1}{2nd Principal Component}\PY{l+s+s1}{\PYZsq{}}\PY{p}{)}
         \PY{n}{ax}\PY{o}{.}\PY{n}{legend}\PY{p}{(}\PY{n}{loc}\PY{o}{=}\PY{l+s+s1}{\PYZsq{}}\PY{l+s+s1}{best}\PY{l+s+s1}{\PYZsq{}}\PY{p}{)}
         \PY{n}{ax}\PY{o}{.}\PY{n}{set\PYZus{}title}\PY{p}{(}\PY{l+s+s1}{\PYZsq{}}\PY{l+s+s1}{sklearn PCA Computed: Top 2 Principal Components}\PY{l+s+se}{\PYZbs{}n}\PY{l+s+s1}{Colored by Class}\PY{l+s+s1}{\PYZsq{}}\PY{p}{)}
         \PY{n}{plt}\PY{o}{.}\PY{n}{show}\PY{p}{(}\PY{p}{)}
\end{Verbatim}


    \begin{center}
    \adjustimage{max size={0.9\linewidth}{0.9\paperheight}}{output_22_0.png}
    \end{center}
    { \hspace*{\fill} \\}
    
    The separation of the classes using PCA is noticeable but not stark.
Also note how the explicitly computed PCA is a mirror image of the PCA
computed with \texttt{sklearn.decomposition.PCA}.

    \subsubsection{Problem 2f (10 points)}\label{problem-2f-10-points}

The PCA that you have just completed takes each data point and projects
it using a weighted sum of features. One could also do the opposite to
map the features as a weighted sum of the data entries. How could this
be done? What is a potential issue? Describe these in a few sentences
(do not code it).

    To recover the "original" data after doing principal component analysis
we can simply reverse the operations. Since the original projection
\(Y\) is a linear combination of the mean adjusted original data \(A\)
and the feature vector \(F\), we can reverse \(Y = FA\) using
\(A = F^TY\). Finally we would account for the mean by adding it back
in. One potential issue of reversing PCA to recover the original data
would be non-linearity in the dataset. Since PCA is a linear
transformation, any non-linearities in the data will be lost when
recovering.

If our goal is to do a projection of features into a lower space
weighted by data points, as was clarified on the forums, then we could
rearrange the equation above to solve for \(F\) such that \(Y=A/F^T\).

    \subsubsection{Problem 3 MDS (10 points)}\label{problem-3-mds-10-points}

For the same data set, repeat 2e using sklearn's Multidimensional
scaling algorithm.

    \begin{Verbatim}[commandchars=\\\{\}]
{\color{incolor}In [{\color{incolor}16}]:} \PY{c+c1}{\PYZsh{} perform PCA using sklearn}
         \PY{n}{mds} \PY{o}{=} \PY{n}{MDS}\PY{p}{(}\PY{n}{n\PYZus{}components}\PY{o}{=}\PY{l+m+mi}{2}\PY{p}{,} \PY{n}{random\PYZus{}state}\PY{o}{=}\PY{l+m+mi}{77}\PY{p}{)}
         \PY{n}{df\PYZus{}mds} \PY{o}{=} \PY{n}{pd}\PY{o}{.}\PY{n}{DataFrame}\PY{p}{(}\PY{n}{mds}\PY{o}{.}\PY{n}{fit\PYZus{}transform}\PY{p}{(}\PY{n}{df}\PY{p}{[}\PY{n}{cols}\PY{p}{]}\PY{p}{)}\PY{p}{)}
         
         \PY{c+c1}{\PYZsh{} add target for plotting}
         \PY{n}{df\PYZus{}mds}\PY{p}{[}\PY{l+s+s1}{\PYZsq{}}\PY{l+s+s1}{target}\PY{l+s+s1}{\PYZsq{}}\PY{p}{]} \PY{o}{=} \PY{n}{df}\PY{p}{[}\PY{l+s+s1}{\PYZsq{}}\PY{l+s+s1}{target}\PY{l+s+s1}{\PYZsq{}}\PY{p}{]}
         
         \PY{c+c1}{\PYZsh{} plot the first two PCs}
         \PY{n}{fig}\PY{p}{,} \PY{n}{ax} \PY{o}{=} \PY{n}{plt}\PY{o}{.}\PY{n}{subplots}\PY{p}{(}\PY{n}{figsize}\PY{o}{=}\PY{p}{(}\PY{l+m+mi}{12}\PY{p}{,} \PY{l+m+mi}{9}\PY{p}{)}\PY{p}{)}
         \PY{n}{i} \PY{o}{=} \PY{l+m+mi}{0}
         \PY{k}{for} \PY{n}{target}\PY{p}{,} \PY{n}{\PYZus{}df} \PY{o+ow}{in} \PY{n}{df\PYZus{}mds}\PY{o}{.}\PY{n}{groupby}\PY{p}{(}\PY{l+s+s1}{\PYZsq{}}\PY{l+s+s1}{target}\PY{l+s+s1}{\PYZsq{}}\PY{p}{)}\PY{p}{:}
             \PY{n}{color} \PY{o}{=} \PY{n}{colors}\PY{p}{[}\PY{n}{i}\PY{p}{]}
             \PY{n}{label} \PY{o}{=} \PY{n}{unq\PYZus{}tgts}\PY{p}{[}\PY{n}{i}\PY{p}{]}
             \PY{n}{ax}\PY{o}{.}\PY{n}{scatter}\PY{p}{(}\PY{n}{\PYZus{}df}\PY{p}{[}\PY{l+m+mi}{0}\PY{p}{]}\PY{p}{,} \PY{n}{\PYZus{}df}\PY{p}{[}\PY{l+m+mi}{1}\PY{p}{]}\PY{p}{,} \PY{n}{color}\PY{o}{=}\PY{n}{color}\PY{p}{,} \PY{n}{label}\PY{o}{=}\PY{n}{label}\PY{p}{,} \PY{n}{alpha}\PY{o}{=}\PY{o}{.}\PY{l+m+mi}{65}\PY{p}{)}
             \PY{n}{i} \PY{o}{+}\PY{o}{=} \PY{l+m+mi}{1}
         \PY{n}{ax}\PY{o}{.}\PY{n}{grid}\PY{p}{(}\PY{n}{alpha}\PY{o}{=}\PY{o}{.}\PY{l+m+mi}{7}\PY{p}{)}
         \PY{n}{sns}\PY{o}{.}\PY{n}{despine}\PY{p}{(}\PY{p}{)}
         \PY{n}{ax}\PY{o}{.}\PY{n}{set\PYZus{}xlabel}\PY{p}{(}\PY{l+s+s1}{\PYZsq{}}\PY{l+s+s1}{1st Dimension}\PY{l+s+s1}{\PYZsq{}}\PY{p}{)}
         \PY{n}{ax}\PY{o}{.}\PY{n}{set\PYZus{}ylabel}\PY{p}{(}\PY{l+s+s1}{\PYZsq{}}\PY{l+s+s1}{2nd Dimension}\PY{l+s+s1}{\PYZsq{}}\PY{p}{)}
         \PY{n}{ax}\PY{o}{.}\PY{n}{legend}\PY{p}{(}\PY{n}{loc}\PY{o}{=}\PY{l+s+s1}{\PYZsq{}}\PY{l+s+s1}{best}\PY{l+s+s1}{\PYZsq{}}\PY{p}{)}
         \PY{n}{ax}\PY{o}{.}\PY{n}{set\PYZus{}title}\PY{p}{(}\PY{l+s+s1}{\PYZsq{}}\PY{l+s+s1}{MDS: Top 2 Scaled Dimensions}\PY{l+s+se}{\PYZbs{}n}\PY{l+s+s1}{Colored by Class}\PY{l+s+s1}{\PYZsq{}}\PY{p}{)}
         \PY{n}{plt}\PY{o}{.}\PY{n}{show}\PY{p}{(}\PY{p}{)}
\end{Verbatim}


    \begin{center}
    \adjustimage{max size={0.9\linewidth}{0.9\paperheight}}{output_27_0.png}
    \end{center}
    { \hspace*{\fill} \\}
    
    The separation of the classes using \texttt{MDS} appears similar to
\texttt{PCA}: groups are similarly overlapping and are similarly far
apart when comparing the two approaches.

    \subsubsection{Problem 4a t-SNE (5
points)}\label{problem-4a-t-sne-5-points}

Repeat 2e using a t-SNE plot with the default settings.

    \begin{Verbatim}[commandchars=\\\{\}]
{\color{incolor}In [{\color{incolor}17}]:} \PY{n+nd}{@timing\PYZus{}function}
         \PY{k}{def} \PY{n+nf}{plot\PYZus{}tsne}\PY{p}{(}\PY{n}{df}\PY{p}{,} \PY{n}{cols}\PY{p}{,} \PY{n}{perplexity}\PY{o}{=}\PY{l+m+mi}{30}\PY{p}{,} \PY{n}{ax}\PY{o}{=}\PY{k+kc}{None}\PY{p}{,} \PY{n}{title}\PY{o}{=}\PY{k+kc}{None}\PY{p}{,} \PY{n}{method}\PY{o}{=}\PY{l+s+s1}{\PYZsq{}}\PY{l+s+s1}{barnes\PYZus{}hut}\PY{l+s+s1}{\PYZsq{}}\PY{p}{,} \PY{n}{random\PYZus{}state}\PY{o}{=}\PY{l+m+mi}{77}\PY{p}{)}\PY{p}{:}
             \PY{l+s+sd}{\PYZsq{}\PYZsq{}\PYZsq{}}
         \PY{l+s+sd}{    DataFrame must have target column.  Value for perplexity }
         \PY{l+s+sd}{    defaults to that for sklearn\PYZsq{}s API. Parameter `cols` must}
         \PY{l+s+sd}{    be numeric names of DataFrame `df`.  }
         \PY{l+s+sd}{    \PYZsq{}\PYZsq{}\PYZsq{}}
             \PY{c+c1}{\PYZsh{} perform TSNE using sklearn}
             \PY{n}{tsne} \PY{o}{=} \PY{n}{TSNE}\PY{p}{(}\PY{n}{n\PYZus{}components}\PY{o}{=}\PY{l+m+mi}{2}\PY{p}{,} 
                         \PY{n}{random\PYZus{}state}\PY{o}{=}\PY{n}{random\PYZus{}state}\PY{p}{,} 
                         \PY{n}{perplexity}\PY{o}{=}\PY{n}{perplexity}\PY{p}{,} 
                         \PY{n}{method}\PY{o}{=}\PY{n}{method}\PY{p}{)}
             \PY{n}{df\PYZus{}tsne} \PY{o}{=} \PY{n}{pd}\PY{o}{.}\PY{n}{DataFrame}\PY{p}{(}\PY{n}{tsne}\PY{o}{.}\PY{n}{fit\PYZus{}transform}\PY{p}{(}\PY{n}{df}\PY{p}{[}\PY{n}{cols}\PY{p}{]}\PY{p}{)}\PY{p}{)}
         
             \PY{c+c1}{\PYZsh{} add target for plotting}
             \PY{n}{df\PYZus{}tsne}\PY{p}{[}\PY{l+s+s1}{\PYZsq{}}\PY{l+s+s1}{target}\PY{l+s+s1}{\PYZsq{}}\PY{p}{]} \PY{o}{=} \PY{n}{df}\PY{p}{[}\PY{l+s+s1}{\PYZsq{}}\PY{l+s+s1}{target}\PY{l+s+s1}{\PYZsq{}}\PY{p}{]}
         
             \PY{c+c1}{\PYZsh{} plot the first two vectors}
             \PY{k}{if} \PY{n}{ax} \PY{o+ow}{is} \PY{k+kc}{None}\PY{p}{:}
                 \PY{n}{fig}\PY{p}{,} \PY{n}{ax} \PY{o}{=} \PY{n}{plt}\PY{o}{.}\PY{n}{subplots}\PY{p}{(}\PY{n}{figsize}\PY{o}{=}\PY{p}{(}\PY{l+m+mi}{12}\PY{p}{,} \PY{l+m+mi}{9}\PY{p}{)}\PY{p}{)}
             \PY{n}{i} \PY{o}{=} \PY{l+m+mi}{0}
             \PY{k}{for} \PY{n}{target}\PY{p}{,} \PY{n}{\PYZus{}df} \PY{o+ow}{in} \PY{n}{df\PYZus{}tsne}\PY{o}{.}\PY{n}{groupby}\PY{p}{(}\PY{l+s+s1}{\PYZsq{}}\PY{l+s+s1}{target}\PY{l+s+s1}{\PYZsq{}}\PY{p}{)}\PY{p}{:}
                 \PY{n}{color} \PY{o}{=} \PY{n}{colors}\PY{p}{[}\PY{n}{i}\PY{p}{]}
                 \PY{n}{label} \PY{o}{=} \PY{n}{unq\PYZus{}tgts}\PY{p}{[}\PY{n}{i}\PY{p}{]}
                 \PY{n}{ax}\PY{o}{.}\PY{n}{scatter}\PY{p}{(}\PY{n}{\PYZus{}df}\PY{p}{[}\PY{l+m+mi}{0}\PY{p}{]}\PY{p}{,} \PY{n}{\PYZus{}df}\PY{p}{[}\PY{l+m+mi}{1}\PY{p}{]}\PY{p}{,} \PY{n}{color}\PY{o}{=}\PY{n}{color}\PY{p}{,} \PY{n}{label}\PY{o}{=}\PY{n}{label}\PY{p}{,} \PY{n}{alpha}\PY{o}{=}\PY{o}{.}\PY{l+m+mi}{65}\PY{p}{)}
                 \PY{n}{i} \PY{o}{+}\PY{o}{=} \PY{l+m+mi}{1}
             \PY{n}{ax}\PY{o}{.}\PY{n}{grid}\PY{p}{(}\PY{n}{alpha}\PY{o}{=}\PY{o}{.}\PY{l+m+mi}{7}\PY{p}{)}
             \PY{n}{sns}\PY{o}{.}\PY{n}{despine}\PY{p}{(}\PY{p}{)}
             \PY{n}{ax}\PY{o}{.}\PY{n}{set\PYZus{}xlabel}\PY{p}{(}\PY{l+s+s1}{\PYZsq{}}\PY{l+s+s1}{1st Dimension}\PY{l+s+s1}{\PYZsq{}}\PY{p}{)}
             \PY{n}{ax}\PY{o}{.}\PY{n}{set\PYZus{}ylabel}\PY{p}{(}\PY{l+s+s1}{\PYZsq{}}\PY{l+s+s1}{2nd Dimension}\PY{l+s+s1}{\PYZsq{}}\PY{p}{)}
             \PY{n}{ax}\PY{o}{.}\PY{n}{legend}\PY{p}{(}\PY{n}{loc}\PY{o}{=}\PY{l+s+s1}{\PYZsq{}}\PY{l+s+s1}{best}\PY{l+s+s1}{\PYZsq{}}\PY{p}{)}
             \PY{k}{if} \PY{n}{title} \PY{o+ow}{is} \PY{k+kc}{None}\PY{p}{:}
                 \PY{n}{ax}\PY{o}{.}\PY{n}{set\PYZus{}title}\PY{p}{(}\PY{l+s+s1}{\PYZsq{}}\PY{l+s+s1}{T\PYZhy{}SNE: Top 2 Scaled Dimensions}\PY{l+s+se}{\PYZbs{}n}\PY{l+s+s1}{Colored by Class}\PY{l+s+s1}{\PYZsq{}}\PY{p}{)}
             \PY{k}{else}\PY{p}{:}
                 \PY{n}{ax}\PY{o}{.}\PY{n}{set\PYZus{}title}\PY{p}{(}\PY{n}{title}\PY{p}{)}
             \PY{k}{if} \PY{n}{ax} \PY{o+ow}{is} \PY{k+kc}{None}\PY{p}{:} 
                 \PY{n}{plt}\PY{o}{.}\PY{n}{show}\PY{p}{(}\PY{p}{)}
             
         \PY{n}{plot\PYZus{}tsne}\PY{p}{(}\PY{n}{df}\PY{p}{,} \PY{n}{cols}\PY{p}{,} \PY{n}{perplexity}\PY{o}{=}\PY{l+m+mi}{30}\PY{p}{)}
\end{Verbatim}


    \begin{Verbatim}[commandchars=\\\{\}]
Runtime: 5.85 seconds

    \end{Verbatim}

    \begin{center}
    \adjustimage{max size={0.9\linewidth}{0.9\paperheight}}{output_30_1.png}
    \end{center}
    { \hspace*{\fill} \\}
    
    T-SNE, even with default settings, clearly does a much better job than
both \texttt{PCA} and \texttt{MDS} of separating classes while still
preserving the intra-group proximity between points.

    \subsubsection{Problem 4b t-SNE perplexity (5
points)}\label{problem-4b-t-sne-perplexity-5-points}

Try out a few t-SNE plots by varying the perplexity. State the best
perplexity for separating the 8 different classes and describe your
rationale in a sentence or two. Report the average calculation time for
the t-SNE projection over a number of iterations.

    \begin{Verbatim}[commandchars=\\\{\}]
{\color{incolor}In [{\color{incolor}18}]:} \PY{c+c1}{\PYZsh{} use \PYZpc{}timemit}
         \PY{n}{perplexities} \PY{o}{=} \PY{p}{[}\PY{l+m+mi}{2}\PY{p}{,} \PY{l+m+mi}{4}\PY{p}{,} \PY{l+m+mi}{8}\PY{p}{,} \PY{l+m+mi}{16}\PY{p}{,} \PY{l+m+mi}{32}\PY{p}{,} \PY{l+m+mi}{64}\PY{p}{,} \PY{l+m+mi}{128}\PY{p}{,} \PY{l+m+mi}{256}\PY{p}{,} \PY{l+m+mi}{512}\PY{p}{,} \PY{l+m+mi}{1024}\PY{p}{]}
         \PY{n}{nrows} \PY{o}{=} \PY{n+nb}{len}\PY{p}{(}\PY{n}{perplexities}\PY{p}{)}\PY{o}{/}\PY{o}{/}\PY{l+m+mi}{2}
         \PY{n}{height} \PY{o}{=} \PY{l+m+mi}{7}\PY{o}{*}\PY{n}{nrows}
         \PY{n}{fig}\PY{p}{,} \PY{n}{axes} \PY{o}{=} \PY{n}{plt}\PY{o}{.}\PY{n}{subplots}\PY{p}{(}\PY{n}{nrows}\PY{p}{,} \PY{l+m+mi}{2}\PY{p}{,} \PY{n}{figsize}\PY{o}{=}\PY{p}{(}\PY{l+m+mi}{17}\PY{p}{,} \PY{n}{height}\PY{p}{)}\PY{p}{)}
         \PY{k}{for} \PY{n}{p}\PY{p}{,} \PY{n}{perplexity} \PY{o+ow}{in} \PY{n+nb}{enumerate}\PY{p}{(}\PY{n}{perplexities}\PY{p}{)}\PY{p}{:}
             \PY{n}{title} \PY{o}{=} \PY{l+s+s1}{\PYZsq{}}\PY{l+s+s1}{T\PYZhy{}SNE: Top 2 Scaled Dimensions Colored by Class}\PY{l+s+se}{\PYZbs{}n}\PY{l+s+s1}{Perplexity = }\PY{l+s+si}{\PYZob{}\PYZcb{}}\PY{l+s+s1}{\PYZsq{}}\PY{o}{.}\PY{n}{format}\PY{p}{(}\PY{n}{perplexity}\PY{p}{)}
             \PY{k}{if} \PY{n}{p} \PY{o}{\PYZlt{}} \PY{n}{nrows}\PY{p}{:}
                 \PY{n}{ax} \PY{o}{=} \PY{n}{axes}\PY{p}{[}\PY{n}{p}\PY{p}{]}\PY{p}{[}\PY{l+m+mi}{0}\PY{p}{]}
             \PY{k}{else}\PY{p}{:}
                 \PY{n}{ax} \PY{o}{=} \PY{n}{axes}\PY{p}{[}\PY{n}{p}\PY{o}{\PYZhy{}}\PY{n}{nrows}\PY{p}{]}\PY{p}{[}\PY{l+m+mi}{1}\PY{p}{]}
             \PY{n+nb}{print}\PY{p}{(}\PY{l+s+s1}{\PYZsq{}}\PY{l+s+s1}{Running t\PYZhy{}SNE @perplexity = }\PY{l+s+si}{\PYZpc{}i}\PY{l+s+s1}{ }\PY{l+s+s1}{\PYZsq{}} \PY{o}{\PYZpc{}}\PY{k}{perplexity})
             \PY{n}{plot\PYZus{}tsne}\PY{p}{(}\PY{n}{df}\PY{p}{,} \PY{n}{cols}\PY{p}{,} \PY{n}{perplexity}\PY{o}{=}\PY{n}{perplexity}\PY{p}{,} \PY{n}{ax}\PY{o}{=}\PY{n}{ax}\PY{p}{,} \PY{n}{title}\PY{o}{=}\PY{n}{title}\PY{p}{)}
             \PY{n+nb}{print}\PY{p}{(}\PY{l+s+s1}{\PYZsq{}}\PY{l+s+s1}{\PYZsq{}}\PY{p}{)}
         \PY{n}{plt}\PY{o}{.}\PY{n}{show}\PY{p}{(}\PY{p}{)}
\end{Verbatim}


    \begin{Verbatim}[commandchars=\\\{\}]
Running t-SNE @perplexity = 2 
Runtime: 2.37 seconds

Running t-SNE @perplexity = 4 
Runtime: 2.5 seconds

Running t-SNE @perplexity = 8 
Runtime: 3.12 seconds

Running t-SNE @perplexity = 16 
Runtime: 4.13 seconds

Running t-SNE @perplexity = 32 
Runtime: 5.64 seconds

Running t-SNE @perplexity = 64 
Runtime: 7.71 seconds

Running t-SNE @perplexity = 128 
Runtime: 9.55 seconds

Running t-SNE @perplexity = 256 
Runtime: 10.71 seconds

Running t-SNE @perplexity = 512 
Runtime: 6.1 seconds

Running t-SNE @perplexity = 1024 
Runtime: 6.0 seconds


    \end{Verbatim}

    \begin{center}
    \adjustimage{max size={0.9\linewidth}{0.9\paperheight}}{output_33_1.png}
    \end{center}
    { \hspace*{\fill} \\}
    
    \begin{Verbatim}[commandchars=\\\{\}]
{\color{incolor}In [{\color{incolor}23}]:} \PY{n}{times} \PY{o}{=} \PY{p}{\PYZob{}}\PY{l+m+mi}{2}\PY{p}{:} \PY{l+m+mf}{2.37}\PY{p}{,} \PY{l+m+mi}{4}\PY{p}{:} \PY{l+m+mf}{2.5}\PY{p}{,} \PY{l+m+mi}{8}\PY{p}{:} \PY{l+m+mf}{3.12}\PY{p}{,} \PY{l+m+mi}{16}\PY{p}{:} \PY{l+m+mf}{4.13}\PY{p}{,} \PY{l+m+mi}{32}\PY{p}{:} \PY{l+m+mf}{5.64}\PY{p}{,} \PY{l+m+mi}{64}\PY{p}{:} \PY{l+m+mf}{7.71}\PY{p}{,} \PY{l+m+mi}{128}\PY{p}{:} \PY{l+m+mf}{9.55}\PY{p}{,} \PY{l+m+mi}{256}\PY{p}{:} \PY{l+m+mf}{10.71}\PY{p}{,} \PY{l+m+mi}{512}\PY{p}{:} \PY{l+m+mf}{6.1}\PY{p}{\PYZcb{}}
         \PY{n}{avg\PYZus{}time} \PY{o}{=} \PY{n}{pd}\PY{o}{.}\PY{n}{Series}\PY{p}{(}\PY{n}{times}\PY{p}{)}\PY{o}{.}\PY{n}{mean}\PY{p}{(}\PY{p}{)}
         \PY{n+nb}{print}\PY{p}{(}\PY{l+s+s1}{\PYZsq{}}\PY{l+s+s1}{Average runtime = }\PY{l+s+si}{\PYZpc{}.4f}\PY{l+s+s1}{ seconds}\PY{l+s+s1}{\PYZsq{}} \PY{o}{\PYZpc{}}\PY{k}{avg\PYZus{}time})
\end{Verbatim}


    \begin{Verbatim}[commandchars=\\\{\}]
Average runtime = 5.7589 seconds

    \end{Verbatim}

    While the approach is imperfect, a \texttt{perplexity} value of 32
appears to have the strongest separation between groups while still
preserving local proximities within groups. This is visually apparent in
the plot on the lower left where \texttt{perplexity=32}: we see that the
red cluster is more dense than at lower \texttt{perplexity} values, yet
the distance from other dissimilar groups is preserved. Beyond
\texttt{perplexity=64} we observe that the clusters first begin to grow
more chaotic before finally conveging into some elliptical-like shape.

    \subsubsection{Problem 4c t-SNE randomization (10
points)}\label{problem-4c-t-sne-randomization-10-points}

The S of t-SNE means stochastic or random, usually as a function of
time. Explore whether you can reproduce the result in 4b through a
second projection and plot.

    \begin{Verbatim}[commandchars=\\\{\}]
{\color{incolor}In [{\color{incolor}19}]:} \PY{n}{perplexities} \PY{o}{=} \PY{p}{[}\PY{l+m+mi}{30} \PY{k}{for} \PY{n}{i} \PY{o+ow}{in} \PY{n+nb}{range}\PY{p}{(}\PY{l+m+mi}{4}\PY{p}{)}\PY{p}{]} \PY{c+c1}{\PYZsh{} use default 4 separate times}
         \PY{n}{nrows} \PY{o}{=} \PY{n+nb}{len}\PY{p}{(}\PY{n}{perplexities}\PY{p}{)}\PY{o}{/}\PY{o}{/}\PY{l+m+mi}{2}
         \PY{n}{height} \PY{o}{=} \PY{l+m+mi}{7}\PY{o}{*}\PY{n}{nrows}
         \PY{n}{fig}\PY{p}{,} \PY{n}{axes} \PY{o}{=} \PY{n}{plt}\PY{o}{.}\PY{n}{subplots}\PY{p}{(}\PY{n}{nrows}\PY{p}{,} \PY{l+m+mi}{2}\PY{p}{,} \PY{n}{figsize}\PY{o}{=}\PY{p}{(}\PY{l+m+mi}{17}\PY{p}{,} \PY{n}{height}\PY{p}{)}\PY{p}{)}
         \PY{k}{for} \PY{n}{p}\PY{p}{,} \PY{n}{perplexity} \PY{o+ow}{in} \PY{n+nb}{enumerate}\PY{p}{(}\PY{n}{perplexities}\PY{p}{)}\PY{p}{:}
             \PY{n}{title} \PY{o}{=} \PY{l+s+s1}{\PYZsq{}}\PY{l+s+s1}{T\PYZhy{}SNE: Top 2 Scaled Dimensions Colored by Class}\PY{l+s+se}{\PYZbs{}n}\PY{l+s+s1}{Perplexity = }\PY{l+s+si}{\PYZob{}\PYZcb{}}\PY{l+s+s1}{\PYZsq{}}\PY{o}{.}\PY{n}{format}\PY{p}{(}\PY{n}{perplexity}\PY{p}{)}
             \PY{k}{if} \PY{n}{p} \PY{o}{\PYZlt{}} \PY{n}{nrows}\PY{p}{:}
                 \PY{n}{ax} \PY{o}{=} \PY{n}{axes}\PY{p}{[}\PY{n}{p}\PY{p}{]}\PY{p}{[}\PY{l+m+mi}{0}\PY{p}{]}
             \PY{k}{else}\PY{p}{:}
                 \PY{n}{ax} \PY{o}{=} \PY{n}{axes}\PY{p}{[}\PY{n}{p}\PY{o}{\PYZhy{}}\PY{n}{nrows}\PY{p}{]}\PY{p}{[}\PY{l+m+mi}{1}\PY{p}{]}
             \PY{n+nb}{print}\PY{p}{(}\PY{l+s+s1}{\PYZsq{}}\PY{l+s+s1}{Running t\PYZhy{}SNE @perplexity = }\PY{l+s+si}{\PYZpc{}i}\PY{l+s+s1}{ }\PY{l+s+s1}{\PYZsq{}} \PY{o}{\PYZpc{}}\PY{k}{perplexity})
             \PY{n}{np}\PY{o}{.}\PY{n}{random}\PY{o}{.}\PY{n}{seed}\PY{p}{(}\PY{n}{p}\PY{p}{)}
             \PY{n}{plot\PYZus{}tsne}\PY{p}{(}\PY{n}{df}\PY{p}{,} \PY{n}{cols}\PY{p}{,} \PY{n}{perplexity}\PY{o}{=}\PY{n}{perplexity}\PY{p}{,} \PY{n}{ax}\PY{o}{=}\PY{n}{ax}\PY{p}{,} \PY{n}{title}\PY{o}{=}\PY{n}{title}\PY{p}{,} \PY{n}{random\PYZus{}state}\PY{o}{=}\PY{n}{p}\PY{p}{)}
             \PY{n+nb}{print}\PY{p}{(}\PY{l+s+s1}{\PYZsq{}}\PY{l+s+s1}{\PYZsq{}}\PY{p}{)}
         \PY{n}{plt}\PY{o}{.}\PY{n}{show}\PY{p}{(}\PY{p}{)}
\end{Verbatim}


    \begin{Verbatim}[commandchars=\\\{\}]
Running t-SNE @perplexity = 30 
Runtime: 6.76 seconds

Running t-SNE @perplexity = 30 
Runtime: 6.62 seconds

Running t-SNE @perplexity = 30 
Runtime: 6.27 seconds

Running t-SNE @perplexity = 30 
Runtime: 6.69 seconds


    \end{Verbatim}

    \begin{center}
    \adjustimage{max size={0.9\linewidth}{0.9\paperheight}}{output_37_1.png}
    \end{center}
    { \hspace*{\fill} \\}
    
    We see above that varying the \texttt{random\_seed} four separate times
yields four unique plots. While the plots are similar (i.e. they
distinguish between groups in a similar way, by preserving distances
between groups adn within) they are clearly different when different
seeds are set. This is due to the fact that stochastic methods are used
to perform t-SNE, and the initial randomization leads to different
results.

    \subsubsection{Problem 4d t-SNE Barnes-Hut (5
points)}\label{problem-4d-t-sne-barnes-hut-5-points}

The default t-SNE method of 4b uses the Barnes-Hut approximation.
Keeping the other parameters the same as 4b, plot the t-SNE result using
the exact method. Which method do you prefer? Compare the average
calculation time for the exact method over a number of iterations.

    \begin{Verbatim}[commandchars=\\\{\}]
{\color{incolor}In [{\color{incolor}24}]:} \PY{n}{perplexities} \PY{o}{=} \PY{p}{[}\PY{l+m+mi}{30} \PY{k}{for} \PY{n}{i} \PY{o+ow}{in} \PY{n+nb}{range}\PY{p}{(}\PY{l+m+mi}{2}\PY{p}{)}\PY{p}{]} \PY{c+c1}{\PYZsh{} use default 2 separate times}
         \PY{n}{methods} \PY{o}{=} \PY{p}{[}\PY{l+s+s1}{\PYZsq{}}\PY{l+s+s1}{barnes\PYZus{}hut}\PY{l+s+s1}{\PYZsq{}}\PY{p}{,} \PY{l+s+s1}{\PYZsq{}}\PY{l+s+s1}{exact}\PY{l+s+s1}{\PYZsq{}}\PY{p}{]}
         \PY{n}{fig}\PY{p}{,} \PY{n}{axes} \PY{o}{=} \PY{n}{plt}\PY{o}{.}\PY{n}{subplots}\PY{p}{(}\PY{l+m+mi}{1}\PY{p}{,} \PY{l+m+mi}{2}\PY{p}{,} \PY{n}{figsize}\PY{o}{=}\PY{p}{(}\PY{l+m+mi}{17}\PY{p}{,} \PY{l+m+mi}{7}\PY{p}{)}\PY{p}{)}
         \PY{k}{for} \PY{n}{p}\PY{p}{,} \PY{n}{perplexity} \PY{o+ow}{in} \PY{n+nb}{enumerate}\PY{p}{(}\PY{n}{perplexities}\PY{p}{)}\PY{p}{:}
             \PY{n}{title} \PY{o}{=} \PY{l+s+s1}{\PYZsq{}}\PY{l+s+s1}{T\PYZhy{}SNE: Top 2 Scaled Dimensions Colored by Class}\PY{l+s+se}{\PYZbs{}n}\PY{l+s+s1}{Method = }\PY{l+s+si}{\PYZob{}\PYZcb{}}\PY{l+s+s1}{\PYZsq{}}\PY{o}{.}\PY{n}{format}\PY{p}{(}\PY{n}{methods}\PY{p}{[}\PY{n}{p}\PY{p}{]}\PY{p}{)}
             \PY{n}{ax} \PY{o}{=} \PY{n}{axes}\PY{p}{[}\PY{n}{p}\PY{p}{]}
             \PY{n+nb}{print}\PY{p}{(}\PY{l+s+s1}{\PYZsq{}}\PY{l+s+s1}{Running t\PYZhy{}SNE @method = }\PY{l+s+si}{\PYZpc{}s}\PY{l+s+s1}{ }\PY{l+s+s1}{\PYZsq{}} \PY{o}{\PYZpc{}}\PY{k}{methods}[p])
             \PY{n}{plot\PYZus{}tsne}\PY{p}{(}\PY{n}{df}\PY{p}{,} \PY{n}{cols}\PY{p}{,} \PY{n}{perplexity}\PY{o}{=}\PY{n}{perplexity}\PY{p}{,} \PY{n}{ax}\PY{o}{=}\PY{n}{ax}\PY{p}{,} \PY{n}{title}\PY{o}{=}\PY{n}{title}\PY{p}{,} \PY{n}{method}\PY{o}{=}\PY{n}{methods}\PY{p}{[}\PY{n}{p}\PY{p}{]}\PY{p}{)}
             \PY{n+nb}{print}\PY{p}{(}\PY{l+s+s1}{\PYZsq{}}\PY{l+s+s1}{\PYZsq{}}\PY{p}{)}
         \PY{n}{plt}\PY{o}{.}\PY{n}{show}\PY{p}{(}\PY{p}{)}
\end{Verbatim}


    \begin{Verbatim}[commandchars=\\\{\}]
Running t-SNE @method = barnes\_hut 
Runtime: 5.69 seconds

Running t-SNE @method = exact 
Runtime: 4.03 seconds


    \end{Verbatim}

    \begin{center}
    \adjustimage{max size={0.9\linewidth}{0.9\paperheight}}{output_40_1.png}
    \end{center}
    { \hspace*{\fill} \\}
    
    \begin{Verbatim}[commandchars=\\\{\}]
{\color{incolor}In [{\color{incolor}27}]:} \PY{k}{def} \PY{n+nf}{run\PYZus{}tsne}\PY{p}{(}\PY{n}{method}\PY{p}{)}\PY{p}{:}
             \PY{c+c1}{\PYZsh{} perform TSNE using sklearn}
             \PY{c+c1}{\PYZsh{} vary method to compare times}
             \PY{n}{tsne} \PY{o}{=} \PY{n}{TSNE}\PY{p}{(}\PY{n}{n\PYZus{}components}\PY{o}{=}\PY{l+m+mi}{2}\PY{p}{,} 
                         \PY{n}{method}\PY{o}{=}\PY{n}{method}\PY{p}{)}
             \PY{n}{df\PYZus{}tsne} \PY{o}{=} \PY{n}{pd}\PY{o}{.}\PY{n}{DataFrame}\PY{p}{(}\PY{n}{tsne}\PY{o}{.}\PY{n}{fit\PYZus{}transform}\PY{p}{(}\PY{n}{df}\PY{p}{[}\PY{n}{cols}\PY{p}{]}\PY{p}{)}\PY{p}{)}
             \PY{k}{return} \PY{k+kc}{None}
         
         \PY{o}{\PYZpc{}}\PY{k}{timeit} run\PYZus{}tsne(\PYZsq{}barnes\PYZus{}hut\PYZsq{})
         \PY{o}{\PYZpc{}}\PY{k}{timeit} run\PYZus{}tsne(\PYZsq{}exact\PYZsq{})
\end{Verbatim}


    \begin{Verbatim}[commandchars=\\\{\}]
5.71 s ± 161 ms per loop (mean ± std. dev. of 7 runs, 1 loop each)
3.93 s ± 65.2 ms per loop (mean ± std. dev. of 7 runs, 1 loop each)

    \end{Verbatim}

    The plots using the two methods appear to highlight the same clusters as
one another, yet the average runtime for the
\texttt{method=\textquotesingle{}exact\textquotesingle{}} approach (3.93
seconds)9 goes faster on average (and with a tighter distribution) than
\texttt{method=\textquotesingle{}barnes\_hut\textquotesingle{}} (5.71
seconds). The exact approach appears to be the better way to go, at
least w.r.t. computation time.

    \subsubsection{How many hours did this homework
take?}\label{how-many-hours-did-this-homework-take}

This will not affect your grade. We will be monitoring time spent on
homework to be sure that we are not over-burdening students.

    First pass, roughly 6 hours. Making sure all is good for submission,
roughly 9 hours.

    \subsubsection{Last step (5 points)}\label{last-step-5-points}

Save this notebook as LastnameFirstnameHW1.ipynb such as
MuskElonHW1.ipynb. Create a pdf of this notebook named similarly. Submit
both the python notebook and the pdf version to the Canvas dropbox. We
require both versions.

    \begin{Verbatim}[commandchars=\\\{\}]
{\color{incolor}In [{\color{incolor}28}]:} \PY{n+nb}{print}\PY{p}{(}\PY{l+s+s1}{\PYZsq{}}\PY{l+s+s1}{OK.}\PY{l+s+s1}{\PYZsq{}}\PY{p}{)}
\end{Verbatim}


    \begin{Verbatim}[commandchars=\\\{\}]
OK.

    \end{Verbatim}


    % Add a bibliography block to the postdoc
    
    
    
    \end{document}
